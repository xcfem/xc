%File: ~/OOP/matrix/Vector.tex
%What: "@(#) Vector.tex, revA"

\noindent {\bf Files}
\indent \#include \f$<\tilde{}\f$/matrix/Vector.h\f$>\f$

\noindent {\bf Class Declaration}
\indent class Vector:

\noindent {\bf Class Hierarchy}
\indent {\bf Vector}

\noindent {\bf Description}
\indent The Vector class provides the vector abstraction. A vector of
//! order \p size is an ordered 1d array of \p size numbers. For
//! example a vector of order 5:

\indent\indent \f$ x = [x_0\f$ \f$x_1\f$ \f$x_2\f$  \f$x_3\f$ \f$x_4]\f$


//! In the Vector class the data is stored in a 1d double array of length
//! equal to the order of the Vector.  At present time none of the methods
//! are declared as being virtual. THIS MAY CHANGE FOR PARALLEL.


\noindent {\bf Public Member Functions }
\indent {\em int Size() const;}
//! Returns the order of the Vector, \p size.

\indent {\em void Zero();}
//! Zeros out the Vector, i.e. sets all the components of the Vector to
\f$0\f$.

\noindent {\bf Overloaded Operator Functions}
\indent {\em double \&operator()(int x) const;}
//! Returns the data at location \p x in the Vector. Assumes (\p x) 
//! is a valid location in the Vector, i.e. \f$0 <= x \f$ order, a
//! segmentation fault or erroneous results can occur if this is not the 
//! case. 

\indent {\em double \&operator()(int x);}
//! Used to set the data at location(\p x) in the Vector. Assumes (\p x)
//! is a valid location in the Vector, i.e. \f$0 <= x < \f$ order, a
//! segmentation fault or erroneous results can occur if this is not the
//! case. 



\indent {\em Vector \&operator-=(const Vector \&V);}
//! A method to subtract the contents of the Vector \p V from the
//! current Vector. If Vectors are not of same order a warning message is
//! printed and nothing is done. 









