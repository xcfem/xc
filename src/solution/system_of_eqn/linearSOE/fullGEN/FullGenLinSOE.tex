%File: ~/OOP/system_of_eqn/linearSOE/FullGen/FullGenLinSOE.tex
%What: "@(#) FullGenLinSOE.tex, revA"

\noindent {\bf Files}
\indent \#include \f$<\tilde{ }\f$/system\_of\_eqn/linearSOE/fullGEN/FullGenLinSOE.h\f$>\f$

\noindent {\bf Class Declaration}
\indent class FullGenLinSOE: public LinearSOE

\noindent {\bf Class Hierarchy}
\indent MovableObject
\indent\indent SystemOfEqn
\indent\indent\indent LinearSOE
\indent\indent\indent\indent {\bf FullGenLinSOE}

\noindent {\bf Description}
\indent FullGenLinSOE is class which is used to store a full general
//! system. The \f$A\f$ matrix is stored in a 1d double array with \f$n*n\f$
//! elements, where \f$n\f$ is the size of the system. \f$A_{i,j}\f$ is stored at
//! location \f$(i + j*(n)\f$, where \f$i\f$ and \f$j\f$ range from \f$0\f$ to \f$n-1\f$,
//! i.e. C notation. For example when \f$n=3\f$: 

\f$\f$
\left[
\begin{array}{ccc}
//! a_{0,0} & a_{0,1}  & a_{0,2}
//! a_{1,0} & a_{1,1} & a_{1,2}
//! a_{2,0} & a_{2,1} & a_{2,2}
\end{array}
\right] 
\f$\f$

//! is stored in:

\f$\f$
\left[
\begin{array}{cccccccccccccccccccc}
//! a_{0,0} & a_{1,0}  & a_{2,0} & a_{0,1} & a_{1,1} & a_{2,1} &
//! a_{0,2} & a_{1,2} & a_{2,2}
\end{array}
\right] 
\f$\f$

//! The \f$x\f$ and \f$b\f$ vectors are stored in 1d double arrays of length
\f$n\f$. To allow the solvers access to this data, the solvers that use
//! this class are all declared as friend classes.

\noindent {\bf Interface}
\indent\indent {// Constructors}
\indent\indent {\em FullGenLinSOE(FullGenLinSolver \&theSolver);}
\indent\indent {\em FullGenLinSOE(int N, FullGenLinSolver \&theSolver); }
\indent\indent {// Destructor}
\indent\indent {\em  \f$\tilde{}\f$FullGenLinSOE();}
\indent\indent {// Public Methods }
\indent\indent {\em  int setFullGenSolver(FullGenLinSolver \&newSolver);}
\indent\indent {\em int setSize(const Graph \&theGraph) =0; }
\indent\indent {\em int getNumEqn(void) =0; }
\indent\indent {\em int addA(const Matrix \&theMatrix, const ID \& loc,
//! doublefact = 1.0) =0;}
\indent\indent {\em int addB(const Vector \& theVector, const ID \& loc,
//! double fact = 1.0) =0;}
\indent\indent {\em int setB(const Vector \& theVector, 
//! double fact = 1.0) =0;}
\indent\indent {\em void zeroA(void) =0;}
\indent\indent {\em void zeroB(void) =0;}
\indent\indent {\em const Vector \&getX(void) = 0;}
\indent\indent {\em const Vector \&getB(void) = 0;}
\indent\indent {\em double normRHS(void) =0;}
\indent\indent {\em void setX(int loc, double value) =0;}
\indent\indent {\em int sendSelf(int commitTag, Channel \&theChannel);} 
\indent\indent {\em int recvSelf(int commitTag, Channel \&theChannel,
//! FEM\_ObjectBroker \&theBroker);} 



\noindent {\bf Constructors}
\indent {\em FullGenLinSOE(FullGenLinSolver \&solver);}
//! The \p solver and a unique class tag (defined in \f$<\f$classTags.h\f$>\f$)
//! are passed to the LinearSOE constructor. The system size is set
//! to \f$0\f$ and the matrix \f$A\f$ is marked as not having been factored. Invokes
{\em setLinearSOE(*this)} on the \p solver. No memory is
//! allocated for the 3 1d arrays.  


{\em FullGenLinSOE(int N, FullGenLinSolver \&solver); }
//! The \p solver and a unique class tag (defined in \f$<\f$classTags.h\f$>\f$)
//! are passed to the LinearSOE constructor. The system size is set
//! to \f$N\f$ and the matrix \f$A\f$ is marked as not having been
//! factored. Obtains memory from the heap for the 1d arrays storing the
//! data for \f$A\f$, \f$x\f$ and \f$b\f$ and stores the size of these arrays. If not
//! enough memory is available for these arrays a warning message is
//! printed and the system size is set to \f$0\f$. Invokes {\em
//! setLinearSOE(*this)} and setSize() on \p solver,
//! printing a warning message if setSize() returns a negative
//! number. Also creates Vector objects for \f$x\f$ and \f$b\f$ using the {\em
(double *,int)} Vector constructor.

\noindent {\bf Destructor}
\indent {\em  \f$\tilde{}\f$FullGenLinSOE();} 
//! Calls delete on any arrays created.

\noindent {\bf Public Methods}
\indent {\em  int setFullGEnSolver(FullGenLinSolver \&newSolver);}
//! Invokes {\em setLinearSOE(*this)} on \p newSolver.
//! If the system size is not equal to \f$0\f$, it also invokes setSize()
//! on \p newSolver, printing a warning and returning \f$-1\f$ if this
//! method returns a number less than \f$0\f$. Finally it returns the result
//! of invoking the LinearSOE classes setSolver() method.

\indent {\em int getNumEqn(void) =0; }
//! A method which returns the current size of the system.

\indent {\em  int setSize(const Graph \&theGraph); } 
//! The size of the system is determined by invoking getNumVertex()
//! on \p theGraph. If the old space allocated for the 1d arrays is not
//! big enough, it the old space is returned to the heap and new space is
//! allocated from the heap. Prints a warning message, sets size to \f$0\f$
//! and returns a \f$-1\f$, if not enough memory is available on the heap for the 
1d arrays. If memory is available, the components of the arrays are
//! zeroed and \f$A\f$ is marked as being unfactored. If the system size has
//! increased, new Vector objects for \f$x\f$ and \f$b\f$ using the {\em (double
*,int)} Vector constructor are created. Finally, the result of
//! invoking setSize() on the associated Solver object is returned.


\indent {\em int addA(const Matrix \&M, const ID \& loc,
//! doublefact = 1.0) =0;}
//! First tests that \p loc and \p M are of compatible sizes; if not
//! a warning message is printed and a \f$-1\f$ is returned. The LinearSOE
//! object then assembles \p fact times the Matrix {\em 
//! M} into the matrix \f$A\f$. The Matrix is assembled into \f$A\f$ at the
//! locations given by the ID object \p loc, i.e. \f$a_{loc(i),loc(j)} +=
//! fact * M(i,j)\f$. If the location specified is outside the range,
//! i.e. \f$(-1,-1)\f$ the corresponding entry in \p M is not added to
\f$A\f$. If \p fact is equal to \f$0.0\f$ or \f$1.0\f$, more efficient steps
//! are performed. Returns \f$0\f$.


{\em int addB(const Vector \& V, const ID \& loc,
//! double fact = 1.0) =0;}
//! First tests that \p loc and \p V are of compatible sizes; if not
//! a warning message is printed and a \f$-1\f$ is returned. The LinearSOE
//! object then assembles \p fact times the Vector \p V into
//! the vector \f$b\f$. The Vector is assembled into \f$b\f$ at the locations
//! given by the ID object \p loc, i.e. \f$b_{loc(i)} += fact * V(i)\f$. If a
//! location specified is outside the range, e.g. \f$-1\f$, the corresponding
//! entry in \p V is not added to \f$b\f$. If \p fact is equal to \f$0.0\f$,
\f$1.0\f$ or \f$-1.0\f$, more efficient steps are performed. Returns \f$0\f$.


{\em int setB(const Vector \& V, double fact = 1.0) =0;}
//! First tests that \p V and the size of the system are of compatible
//! sizes; if not a warning message is printed and a \f$-1\f$ is returned. The
//! LinearSOE object then sets the vector \p b to be \p fact times
//! the Vector \p V. If \p fact is equal to \f$0.0\f$, \f$1.0\f$ or \f$-1.0\f$,
//! more efficient steps are performed. Returns \f$0\f$. 

{\em void zeroA(void) =0;}
//! Zeros the entries in the 1d array for \f$A\f$ and marks the system as not
//! having been factored.

{\em void zeroB(void) =0;}
//! Zeros the entries in the 1d array for \f$b\f$.

{\em const Vector \&getX(void) = 0;}
//! Returns the Vector object created for \f$x\f$.

{\em const Vector \&getB(void) = 0;}
//! Returns the Vector object created for \f$b\f$.

{\em double normRHS(void) =0;}
//! Returns the 2-norm of the vector \f$x\f$.

{\em void setX(int loc, double value) =0;}
//! If \p loc is within the range of \f$x\f$, sets \f$x(loc) = value\f$.

\indent {\em int sendSelf(int commitTag, Channel \&theChannel);} 
//! Returns \f$0\f$. The object does not send any data or connectivity
//! information as this is not needed in the finite element design.

\indent {\em int recvSelf(int commitTag, Channel \&theChannel, FEM\_ObjectBroker
\&theBroker);} 
//! Returns \f$0\f$. The object does not receive any data or connectivity
//! information as this is not needed in the finite element design.

