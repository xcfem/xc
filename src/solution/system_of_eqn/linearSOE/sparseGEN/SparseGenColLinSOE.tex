%File: ~/OOP/system_of_eqn/linearSOE/sparseGen/SparseGenColLinSOE.tex
%What: "@(#) SparseGenColLinSOE.tex, revA"

\noindent {\bf Files}
\indent \#include \f$<\tilde{ }\f$/system\_of\_eqn/linearSOE/sparseGen/SparseGenColLinSOE.h\f$>\f$

\noindent {\bf Class Decleration}
\indent class SparseGenColLinSOE: public LinearSOE

\noindent {\bf Class Hierarchy}
\indent MovableObject
\indent\indent SystemOfEqn
\indent\indent\indent LinearSOE
\indent\indent\indent\indent {\bf SparseGenColLinSOE}

\noindent {\bf Description}
\indent SparseGenColLinSOE is class which is used to store the matrix
//! equation \f$Ax=b\f$ of order \f$size\f$ using a sparse column-compacted storage
//! scheme for \f$A\f$. The \f$A\f$ matrix is stored in a 1d double array with
\f$nnz\f$ elements, where \f$nnz\f$ is the number of non-zeroes in the matrix
\f$A\f$. Two additional 1d integer arrays \f$rowA\f$ and \f$colStartA\f$ are used to
//! store information about the location of the coefficients, with \f$colStartA(i)\f$
//! storing the location in the 1d double array of the start of column \f$i\f$
//! and \f$rowA(j)\f$ identifying the row in \f$A\f$ to which the
\f$j'th\f$ component in the 1d double array. \f$colStartA\f$ is of
//! dimension \f$size+1\f$ and \f$rowA\f$ of dimension \f$nnz\f$. For example

\f$\f$
\left[
\begin{array}{ccccc}
//! a_{0,0} & 0 & a_{0,2}  & a_{0,3} & 0
//! a_{1,0} & a_{1,1} & 0 & 0 & 0
0 & a_{2,1} & a_{2,2} & 0 & 0
0 & 0 & 0 & a_{3,3} & a_{3,4}
//! a_{4,0} & a_{4,1} & 0 & 0 & a_{4,4}
\end{array}
\right] 
\f$\f$

//! is stored in:

\f$\f$
\left[
\begin{array}{cccccccccccccc}
//! a_{0,0} & a_{1,0}  & a_{4,0} & a_{1,1} & a_{2,1} & a_{4,1} &
//! a_{0,2} & a_{2,2} & a_{0,3} & a_{3,3} & a_{3,4} & a_{4,4}
\end{array}
\right] 
\f$\f$

//! with

\f$\f$
//! colStartA =
\left[
\begin{array}{cccccccccccccc}
0 & 3 & 6 & 8 & 10 & 12
\end{array}
\right] 
\f$\f$

//! and

\f$\f$
//! rowA =
\left[
\begin{array}{cccccccccccccc}
0 & 1 & 4 & 1 & 2 & 4 & 0 & 2 & 0 & 3 & 3 & 4 
\end{array}
\right] 
\f$\f$
//! The \f$x\f$ and \f$b\f$ vectors are stored in 1d double arrays of length \f$n\f$.

\noindent {\bf Interface}
\indent\indent {// Constructors}
\indent\indent {\em SparseGenColLinSOE(SparseGenColLinSolver \&theSolver);}
\indent\indent {\em SparseGenColLinSOE(int N, int NNZ, int *colStartA,
//! int *rowA, SparseGenColLinSolver \&theSolver); }
\indent\indent {// Destructor}
\indent\indent {\em  \f$\tilde{}\f$SparseGenColLinSOE();}
\indent\indent {// Public Methods }
\indent\indent {\em  int setSparseGenSolver(SparseGenColLinSolver \&newSolver);}
\indent\indent {\em int setSize(const Graph \&theGraph) =0; }
\indent\indent {\em int getNumEqn(void) =0; }
\indent\indent {\em int addA(const Matrix \&theMatrix, const ID \& loc,
//! doublefact = 1.0) =0;}
\indent\indent {\em int addB(const Vector \& theVector, const ID \& loc,
//! double fact = 1.0) =0;}
\indent\indent {\em int setB(const Vector \& theVector, 
//! double fact = 1.0) =0;}
\indent\indent {\em void zeroA(void) =0;}
\indent\indent {\em void zeroB(void) =0;}
\indent\indent {\em const Vector \&getX(void) = 0;}
\indent\indent {\em const Vector \&getB(void) = 0;}
\indent\indent {\em double normRHS(void) =0;}
\indent\indent {\em void setX(int loc, double value) =0;}
\indent\indent {\em int sendSelf(int commitTag, Channel \&theChannel);} 
\indent\indent {\em int recvSelf(int commitTag, Channel \&theChannel,
//! FEM\_ObjectBroker \&theBroker);} 


\noindent {\bf Constructors}
\indent {\em SparseGenColLinSOE(SparseGenColLinSolver \&solver);}
//! The \p solver and a unique class tag (defined in \f$<\f$classTags.h\f$>\f$)
//! are passed to the LinearSOE constructor. The system size is set
//! to \f$0\f$ and the matrix \f$A\f$ is marked as not having been factored. Invokes
{\em setLinearSOE(*this)} on the \p solver. No memory is
//! allocated for the 3 1d arrays.  


{\em SparseGenColLinSOE(int N, int NNZ, int *colStartA,
//! int *rowA, SparseGenColLinSolver \&theSolver); }
//! The \p solver and a unique class tag (defined in \f$<\f$classTags.h\f$>\f$)
//! are passed to the LinearSOE constructor. The system size is set
//! to \f$N\f$, the number of non-zeros is set to \f$NNZ\f$ and the matrix \f$A\f$ is
//! marked as not having been factored. Obtains memory from the heap for
//! the 1d arrays storing the data for \f$A\f$, \f$x\f$ and \f$b\f$ and stores the
//! size of these arrays. If not enough memory is available for these
//! arrays a warning message is printed and the system size is set to
\f$0\f$. Invokes setLinearSOE(*this)} and {\em setSize() on \p solver,
//! printing a warning message if setSize() returns a negative
//! number. Also creates Vector objects for \f$x\f$ and \f$b\f$ using the {\em
(double *,int)} Vector constructor. It is up to the user to ensure
//! that \p colStartA and \p rowA are of the correct size and
//! contain the correct data.

\noindent {\bf Destructor}
\indent {\em  \f$\tilde{}\f$SparseGenColLinSOE();} 
//! Calls delete on any arrays created.

\noindent {\bf Public Methods}
\indent {\em  int setSolver(SparseGenColLinSolver \&newSolver);}
//! Invokes {\em setLinearSOE(*this)} on \p newSolver.
//! If the system size is not equal to \f$0\f$, it also invokes setSize()
//! on \p newSolver, printing a warning and returning \f$-1\f$ if this
//! method returns a number less than \f$0\f$. Finally it returns the result
//! of invoking the LinearSOE classes setSolver() method.

\indent {\em int getNumEqn(void) =0; }
//! A method which returns the current size of the system.

\indent {\em  int setSize(const Graph \&theGraph); } 
//! The size of the system is determined from the Graph object {\em
//! theGraph}, which must contain \p size vertices labelled \f$0\f$ through
\f$size-1\f$, the adjacency list for each vertex containing the associated
//! vertices in a column of the matrix \f$A\f$. The size is determined by
//! invoking getNumVertex() on \p theGraph and the number of
//! non-zeros is determined by looking at the size of the adjacenecy list
//! of each vertex in the graph, allowing space for the diagonal term. If
//! the old space allocated for the 1d arrays is not big enough, it the
//! old space is returned to the heap and new space is allocated from the
//! heap. Prints a warning message, sets size to \f$0\f$ and returns a \f$-1\f$,
//! if not enough memory is available on the heap for the 1d arrays. If
//! memory is available, the components of the arrays are 
//! zeroed and \f$A\f$ is marked as being unfactored. If the system size has
//! increased, new Vector objects for \f$x\f$ and \f$b\f$ using the {\em (double
*,int)} Vector constructor are created. The \f$colStartA\f$ and \f$rowA\f$ are
//! then determined by looping through the vertices, setting \f$colStartA(i)
= colStartA(i-1) + 1 + \f$the size of Vertices \f$i\f$ adjacency list and 
//! placing the contents of \f$i\f$ and the adjacency list into \f$rowA\f$ in
//! ascending order. Finally, the result of invoking setSize() on
//! the associated Solver object is returned. 


\indent {\em int addA(const Matrix \&M, const ID \& loc,
//! doublefact = 1.0) =0;}
//! First tests that \p loc and \p M are of compatable sizes; if not
//! a warning message is printed and a \f$-1\f$ is returned. The LinearSOE
//! object then assembles \p fact times the Matrix {\em 
//! M} into the matrix \f$A\f$. The Matrix is assembled into \f$A\f$ at the
//! locations given by the ID object \p loc, i.e. \f$a_{loc(i),loc(j)} +=
//! fact * M(i,j)\f$. If the location specified is outside the range,
//! i.e. \f$(-1,-1)\f$ the corrseponding entry in \p M is not added to
\f$A\f$. If \p fact is equal to \f$0.0\f$ or \f$1.0\f$, more efficient steps
//! are performed. Returns \f$0\f$.


{\em int addB(const Vector \& V, const ID \& loc,
//! double fact = 1.0) =0;}
//! First tests that \p loc and \p V are of compatable sizes; if not
//! a warning message is printed and a \f$-1\f$ is returned. The LinearSOE
//! object then assembles \p fact times the Vector \p V into
//! the vector \f$b\f$. The Vector is assembled into \f$b\f$ at the locations
//! given by the ID object \p loc, i.e. \f$b_{loc(i)} += fact * V(i)\f$. If a
//! location specified is outside the range, e.g. \f$-1\f$, the corresponding
//! entry in \p V is not added to \f$b\f$. If \p fact is equal to \f$0.0\f$,
\f$1.0\f$ or \f$-1.0\f$, more efficient steps are performed. Returns \f$0\f$.


{\em int setB(const Vector \& V, double fact = 1.0) =0;}
//! First tests that \p V and the size of the system are of compatable
//! sizes; if not a warning message is printed and a \f$-1\f$ is returned. The
//! LinearSOE object then sets the vector \p b to be \p fact times
//! the Vector \p V. If \p fact is equal to \f$0.0\f$, \f$1.0\f$ or \f$-1.0\f$,
//! more efficient steps are performed. Returns \f$0\f$. 

{\em void zeroA(void) =0;}
//! Zeros the entries in the 1d array for \f$A\f$ and marks the system as not
//! having been factored.

{\em void zeroB(void) =0;}
//! Zeros the entries in the 1d array for \f$b\f$.

{\em const Vector \&getX(void) = 0;}
//! Returns the Vector object created for \f$x\f$.

{\em const Vector \&getB(void) = 0;}
//! Returns the Vector object created for \f$b\f$.

{\em double normRHS(void) =0;}
//! Returns the 2-norm of the vector \f$x\f$.

{\em void setX(int loc, double value) =0;}
//! If \p loc is within the range of \f$x\f$, sets \f$x(loc) = value\f$.


\indent {\em int sendSelf(int commitTag, Channel \&theChannel);} 
//! Returns \f$0\f$. The object does not send any data or connectivity
//! information as this is not needed in the finite element design.

\indent {\em int recvSelf(int commitTag, Channel \&theChannel, FEM\_ObjectBroker
\&theBroker);} 
//! Returns \f$0\f$. The object does not receive any data or connectivity
//! information as this is not needed in the finite element design.

