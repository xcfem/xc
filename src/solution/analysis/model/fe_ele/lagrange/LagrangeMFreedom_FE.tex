% File: ~/OOP/analysis/model/fe_ele/lagrange/LagrangeMFreedom_FE.tex 
%What: "@(#) LagrangeMFreedom_FE.tex, revA"

\noindent {\bf Files}
\indent \#include \f$<\tilde{ }\f$/analysis/fe\_ele/lagrange/LagrangeMP\_FE.h\f$>\f$

\noindent {\bf Class Declaration}
\indent class LagrangeMP\_FE: public FE\_Element ;

\noindent {\bf Class Hierarchy}
\indent FE\_Element
\indent\indent {\bf LagrangeMP\_FE} 

\noindent {\bf Description}
\indent LagrangeMP\_FE is a subclass of FE\_Element used to enforce a
//! multi point constraint, of the form \f$\U_c = \C_{cr} \U_r\f$, where \f$\U_c\f$ are
//! the constrained degrees-of-freedom at the constrained node, \f$\U_r\f$ are
//! the retained degrees-of-freedom at the retained node and \f$\C_{cr}\f$ a
//! matrix defining the relationship between these degrees-of-freedom. 

//! To enforce the constraint the following are added to the tangent and
//! the residual:
\[ \left[ \begin{array}{cc} 0 & \alpha\C^t \alpha\C & 0 \end{array}
\right] ,
\left\{ \begin{array}{c} 0 0 \end{array} \right\} \]
\noindent 
\noindent at the locations
//! corresponding to the constrained degree-of-freedoms specified by the
//! MP\_Constraint, i.e. \f$[\U_c\f$ \f$\U_r]\f$, and the lagrange multiplier
//! degrees-of-freedom introduced by the LagrangeConstraintHandler for
//! this constraint, \f$\C = [-\I\f$ \f$\C_{cr}]\f$. Nothing is added to the residual.  


\noindent {\bf Class Interface}
\indent // Constructor
\indent {\em LagrangeMP\_FE(Domain \&theDomain, MP\_Constraint
\&theMP, double alpha);}
\indent // Destructor
\indent {\em virtual~ \f$\tilde{}\f$LagrangeMP\_FE();}
\indent // Public Methods
\indent {\em virtual void setID(void);} 
\indent {\em virtual const Matrix \&getTangent(Integrator
*theIntegrator);}  
\indent {\em virtual const Vector \&getResidual(Integrator
*theIntegrator);}  
\indent {\em virtual const Vector \&getTangForce(const Vector
\&disp, double fact = 1.0);    }

\noindent {\bf Constructor}
\indent {\em LagrangeMP\_FE(Domain \&theDomain, MP\_Constraint \&theMP,
//! double alpha);}
//! To construct a LagrangeMP\_FE element to enforce the constraint
//! specified by the MP\_Constraint \p theMP using a default value for
\f$\alpha\f$ of \f$alpha\f$. The FE\_Element class constructor is called with
//! the integers \f$3\f$ and the two times the size of the \p retainedID
//! plus the size of the \p constrainedID at the MP\_Constraint {\em
//! theMP} plus . A Matrix and a Vector object are created for adding the
//! contributions to the tangent and the residual. The residual is
//! zeroed. If the
//! MP\_Constraint is not time varying, then the contribution to the
//! tangent is determined. Links are set to the retained and constrained
//! nodes. The DOF\_Group tag ID is set using the tag of the constrained
//! Nodes DOF\_Group, the tag of the retained Node Dof\_group and the tag
//! of the LagrangeDOF\_Group, \p theGroup. A warning message is printed and 
//! the program is terminated if either not enough memory is available for
//! the Matrices and Vector or the constrained and retained Nodes of their
//! DOF\_Groups do not exist.



\noindent {\bf Destructor}
\indent {\em virtual~ \f$\tilde{}\f$LagrangeMP\_FE();}
//! Invokes delete on any Matrix or Vector objects created in the
//! constructor that have not yet been destroyed.

\noindent {\bf Public Methods}
\indent {\em virtual void setID(void);}
//! Causes the LagrangeMP\_FE to determine the mapping between it's equation
//! numbers and the degrees-of-freedom. This information is obtained by
//! using the mapping information at the DOF\_Group objects associated with
//! the constrained and retained nodes and the LagrangeDOF\_Group, {\em
//! theGroup}. Returns \f$0\f$ if
//! successful. Prints a warning message and returns a negative number if
//! an error occurs: \f$-2\f$ if the
//! Node has no associated DOF\_Group, \f$-3\f$ if the constrained DOF
//! specified is invalid for this Node (sets corresponding ID component to
\f$-1\f$ so nothing is added to the tangent) and \f$-4\f$ if the ID in the
//! DOF\_Group is too small for the Node (again setting corresponding ID
//! component to \f$-1\f$). 

\indent {\em virtual const Matrix \&getTangent(Integrator *theIntegrator);}
//! If the MP\_Constraint is time-varying, from the MP\_Constraint
\p theMP it obtains the current \f$C_{cr}\f$ matrix; it then adds the
//! contribution to the tangent matrix. Returns this tangent Matrix.

\indent {\em virtual const Vector \&getResidual(Integrator *theIntegrator);}
//! Returns the residual, a \f$\zero\f$ Vector.

{\em virtual const Vector \&getTangForce(const Vector \&disp, double
//! fact = 1.0);    }
//! CURRENTLY just returns the \f$0\f$ residual. THIS WILL NEED TO CHANGE FOR
//! ELE-BY-ELE SOLVERS. 

