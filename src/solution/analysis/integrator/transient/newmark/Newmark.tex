%File: ~/OOP/analysis/integrator/Newmark.tex
%What: "@(#) Newmark.tex, revA"

\noindent {\bf Files}
\indent \#include \f$<\tilde{ }\f$/analysis/integrator/Newmark.h\f$>\f$

\noindent {\bf Class Declaration}
\indent class Newmark: public TransientIntegrator

\noindent {\bf Class Hierarchy}
\indent MovableObject
\indent\indent Integrator
\indent\indent\indent IncrementalIntegrator
\indent\indent\indent\indent TransientIntegrator
\indent\indent\indent\indent\indent {\bf Newmark}

\noindent {\bf Description} 
\indent Newmark is a subclass of TransientIntegrator which implements
//! the Newmark method. In the Newmark method, to determine the
//! velocities, accelerations and displacements at time \f$t + \Delta t\f$,
//! the equilibrium equation (expressed for the TransientIntegrator) is
//! typically solved at time \f$t + \Delta t\f$ for \f$\U_{t+\Delta t}\f$,
//! i.e. solve: 

\[ \R (\U_{t + \Delta t}) = \P(t + \Delta t) - \F_I(\Udd_{t+\Delta t})
- \F_R(\Ud_{t + \Delta t},\U_{t + \Delta t}) \]

\noindent for \f$\U_{t+\Delta t}\f$. The following difference relations
//! are used to relate \f$\Ud_{t + \Delta t}\f$ and \f$\Udd_{t + \Delta t}\f$ to
\f$\U_{t + \Delta t}\f$ and the response quantities at time \f$t\f$:

\[
\dot \U_{t + \Delta t} = \frac{\gamma}{\beta \Delta t}
\left( \U_{t + \Delta t} - \U_t \right)
 + \left( 1 - \frac{\gamma}{\beta}\right) \dot \U_t + \Delta t \left(1
- \frac{\gamma}{2 \beta}\right) \ddot \U_t 
\]

\[
\ddot \U_{t + \Delta t} = \frac{1}{\beta {\Delta t}^2}
\left( \U_{t+\Delta t} - \U_t \right)
 - \frac{1}{\beta \Delta t} \dot \U_t + \left( 1 - \frac{1}{2
\beta} \right) \ddot \U_t 
\]

\noindent which  results in the following 

\[ \left[ \frac{1}{\beta \Delta t^2} \M + \frac{\gamma}{\beta \Delta t}
\C + \K \right] \Delta \U_{t + \Delta t}^{(i)} = \P(t + \Delta t) -
\F_I\left(\Udd_{t+\Delta  t}^{(i-1)}\right)
- \F_R\left(\Ud_{t + \Delta t}^{(i-1)},\U_{t + \Delta t}^{(i-1)}\right) \]

\noindent An alternative approach, which does not involve \f$\Delta t\f$
//! in the denumerator (useful for impulse problems), is to solve for the
//! accelerations at time \f$t + \Delta t\f$ 

\[ \R (\Udd_{t + \Delta t}) = \P(t + \Delta t) - \F_I(\Udd_{t+\Delta t})
- \F_R(\Ud_{t + \Delta t},\U_{t + \Delta t}) \]

\noindent where we use following functions to relate \f$\U_{t + \Delta
//! t}\f$ and \f$\Ud_{t + \Delta t}\f$ to \f$\Udd_{t + \Delta t}\f$ and the response
//! quantities at time \f$t\f$:

\[
\U_{t + \Delta t} = \U_t + \Delta t \Ud_t + \left[
\left(\frac{1}{2} - \beta\right)\Udd_t + \beta \Udd_{t + \Delta
//! t}\right] \Delta t^2
\]

\[
\Ud_{t + \Delta t} = \Ud_t + \left[ \left(1 - \gamma\right)\Udd_t +
\gamma\Udd_{t + \Delta t}\right] \Delta t
\]

\noindent which results in the following 

\[ \left[ \M + \gamma \Delta t \C + \beta \Delta t^2 \K \right] \Delta
\Udd_{t + \Delta t}^{(i)} = \P(t + \Delta t) - \F_I\left(\Udd_{t+\Delta 
//! t}^{(i-1)}\right)
- \F_R\left(\Ud_{t + \Delta t}^{(i-1)},\U_{t + \Delta
//! t}^{(i-1)}\right) \]


\pagebreak
\noindent {\bf Class Interface}
\indent // Constructors
\indent {\em Newmark(bool dispFlag = true);} 
\indent {\em Newmark(double gamma, double beta, bool dispFlag = true);} 
\indent {\em Newmark(double gamma, double beta, double alphaM, double
//! betaK, bool dispFlag = true);}
\indent // Destructor
\indent {\em virtual~ \f$\tilde{}\f$Newmark();}
\indent // Public Methods
\indent {\em int formEleTangent(FE\_Element *theEle);}
\indent {\em int formNodTangent(DOF\_Group *theDof);}
\indent {\em int domainChanged(void);}
\indent {\em int newStep(double deltaT);}
\indent {\em int update(const Vector \&\f$\Delta U\f$);}
\indent // Public Methods for Output
\indent {\em int sendSelf(int commitTag, Channel \&theChannel);} 
\indent {\em int recvSelf(int commitTag, Channel \&theChannel,
//! FEM\_ObjectBroker \&theBroker);} 
\indent {\em int Print(OPS\_Stream \&s, int flag = 0);}


\noindent {\bf Constructors}
\indent {\em Newmark(bool dispFlag = true);} 
//! Sets \f$\gamma\f$ to \f$1/2\f$ and \f$\beta\f$ to \f$1/4\f$. Sets a flag indicating
//! whether the incremental solution is done in terms of displacement,
\f$\Delta \U\f$, if \p dispFlag is \p true, or  
//! acceleration, \f$\Delta \ddot \U\f$, if \p dispFlag is \p false. In
//! addition, a flag is set indicating that Rayleigh damping will not be used.


\indent {\em Newmark(double gamma, double beta, bool dispFlag = true);} 
//! Sets \f$\gamma\f$ to \p gamma and \f$\beta\f$ to \p beta. Sets a flag
//! indicating whether the incremental solution is done in terms of
//! displacement or acceleration to \p dispFlag and a flag indicating
//! that Rayleigh damping will not be used.


\indent {\em Newmark(double gamma, double beta, double alphaM, double
//! betaK, bool dispFlag = true);} 
//! This constructor is invoked if Rayleigh damping is to be used, 
//! i.e. \f$\D = \alpha_M M + \beta_K K\f$. 
//! Sets \f$\gamma\f$ to \p gamma, \f$\beta\f$ to \p beta, \f$\alpha_M\f$ to
\p alphaM and \f$\beta_K\f$ to \p betaK. Sets a flag indicating whether the
//! incremental solution is done in terms of displacement or acceleration
//! to \p dispFlag and a flag indicating that Rayleigh damping will 
//! be used. 

\noindent {\bf Destructor}
\indent {\em virtual~ \f$\tilde{}\f$Newmark();} 
//! Invokes the destructor on the Vector objects created.

\noindent {\bf Public Methods}
\indent {\em int formEleTangent(FE\_Element *theEle);}
//! This tangent for each FE\_Element is defined to be \f$\K_e = c1 \K + c2
\D + c3 \M\f$, where c1,c2 and c3 were determined in the last invocation
//! of the newStep() method.  The method returns \f$0\f$ after
//! performing the following operations:
\begin{tabbing}
//! while \= \+ while \= while \= \kill
//! if (RayleighDamping == false) \{ \+
//! theEle-\f$>\f$zeroTang()
//! theEle-\f$>\f$addKtoTang(c1)
//! theEle-\f$>\f$addCtoTang(c2)
//! theEle-\f$>\f$addMtoTang(c3)  \-
\} else \{ \+
//! theEle-\f$>\f$zeroTang()
//! theEle-\f$>\f$addKtoTang(c1 + c2 * \f$\beta_K\f$)
//! theEle-\f$>\f$addMtoTang(c3 + c2 * \f$\alpha_M\f$)  \- 
\}
\end{tabbing}



{\em int formNodTangent(DOF\_Group *theDof);}
//! The method returns \f$0\f$ after performing the following operations:
\begin{tabbing}
//! while \= \+ while \= while \= \kill
//! theDof-\f$>\f$zeroUnbalance()
//! if (RayleighDamping == false)  \+
//! theDof-\f$>\f$addMtoTang(c3)  \-
//! else \+
//! theDof-\f$>\f$addMtoTang(c3 + c2 * \f$\alpha_M\f$)  \- 
\end{tabbing}


{\em int domainChanged(void);}
//! If the size of the LinearSOE has changed, the object deletes any old Vectors
//! created and then creates \f$6\f$ new Vector objects of size equal to {\em
//! theLinearSOE-\f$>\f$getNumEqn()}. There is a Vector object created to store
//! the current displacement, velocity and accelerations at times \f$t\f$ and
\f$t + \Delta t\f$. The response quantities at time \f$t + \Delta t\f$ are
//! then set by iterating over the DOF\_Group objects in the model and
//! obtaining their committed values. 
//! Returns \f$0\f$ if successful, otherwise a warning message and a negative
//! number is returned: \f$-1\f$ if no memory was available for constructing
//! the Vectors.

{\em int newStep(double \f$\Delta t\f$);}
//! The following are performed when this method is invoked:
\begin{enumerate}
\item First sets the values of the three constants {\em c1}, {\em c2}
//! and {\em c3} depending on the flag indicating whether incremental
//! displacements or accelerations are being solved for at each iteration.
//! If \p dispFlag was \p true, {\em c1} is set to \f$1.0\f$, {\em c2} to \f$
\gamma / (\beta \Delta t)\f$ and {\em c3} to \f$1/ (\beta \Delta t^2)\f$. If
//! the flag is \p false {\em c1} is set to \f$\beta \Delta t^2\f$, {\em c2} to \f$
\gamma \Delta t\f$ and {\em c3} to \f$1.0\f$. 
\item Then the Vectors for response quantities at time \f$t\f$ are set
//! equal to those at time \f$t + \Delta t\f$.
\begin{tabbing}
//! while \= while \= while \= while \= \kill
\>\> \f$ \U_t = \U_{t + \Delta t}\f$
\>\> \f$ \Ud_t = \Ud_{t + \Delta t} \f$
\>\> \f$ \Udd_t = \Udd_{t + \Delta t} \f$ 
\end{tabbing}
\item Then the velocity and accelerations approximations at time \f$t +
\Delta t\f$ are set using the difference approximations if {\em
//! dispFlag} was \p true. (displacement and velocity if \p false).
\begin{tabbing}
//! while \= while \= while \= while \= \kill
\>\> if (displIncr == true) \{
\>\>\> \f$ \dot \U_{t + \Delta t} = 
 \left( 1 - \frac{\gamma}{\beta}\right) \dot \U_t + \Delta t \left(1
- \frac{\gamma}{2 \beta}\right) \ddot \U_t \f$
\>\>\> \f$ \ddot \U_{t + \Delta t} = 
 - \frac{1}{\beta \Delta t} \dot \U_t + \left( 1 - \frac{1}{2
\beta} \right) \ddot \U_t  \f$
\>\> \} else \{
\>\>\> \f$ \U_{t + \Delta t} = \U_t + \Delta t \Ud_t + \frac{\Delta
//! t^2}{2}\Udd_t\f$
\>\>\> \f$ \Ud_{t + \Delta t} = \Ud_t +  \Delta t \Udd_t \f$
\>\> \} 
\end{tabbing}
\item The response quantities at the DOF\_Group objects are updated
//! with the new approximations by invoking setResponse() on the
//! AnalysisModel with new quantities for time \f$t + \Delta t\f$.
\begin{tabbing}
//! while \= while \= while \= while \= \kill
\>\> theModel-\f$>\f$setResponse\f$(\U_{t + \Delta t}, \Ud_{t+\Delta t},
\Udd_{t+\Delta t})\f$ 
\end{tabbing}
\item current time is obtained from the AnalysisModel, incremented by
\f$\Delta t\f$, and {\em applyLoad(time, 1.0)} is invoked on the
//! AnalysisModel. 
\item Finally updateDomain() is invoked on the AnalysisModel.
\end{enumerate}
//! The method returns \f$0\f$ if successful, otherwise a negative number is
//! returned: \f$-1\f$ if \f$\gamma\f$ or \f$\beta\f$ are \f$0\f$, \f$-2\f$ if \p dispFlag
//! was true and \f$\Delta t\f$ is \f$0\f$, and \f$-3\f$ if domainChanged()
//! failed or has not been called.

{\em int update(const Vector \&\f$\Delta U\f$);}
//! Invoked this causes the object to increment the DOF\_Group
//! response quantities at time \f$t + \Delta t\f$. The displacement Vector is  
//! incremented by \f$ c1 * \Delta U\f$, the velocity Vector by \f$
//! c2 * \Delta U\f$, and the acceleration Vector by \f$c3 * \Delta U\f$.
//! The response at the DOF\_Group objects are then updated by invoking
//! setResponse() on the AnalysisModel with quantities at time \f$t +
\Delta t\f$. Finally updateDomain() is invoked on the 
//! AnalysisModel. 
\begin{tabbing}
//! while \= while \= while \= while \= \kill
\>\> if (displIncr == true) \{
\>\>\> \f$ \U_{t + \Delta t} += \Delta \U\f$
\>\>\> \f$ \dot \U_{t + \Delta t} += \frac{\gamma}{\beta \Delta t} \Delta \U \f$
\>\>\> \f$ \ddot \U_{t + \Delta t} += \frac{1}{\beta {\Delta t}^2} \Delta
\U \f$
\>\> \} else \{
\>\>\> \f$ \Udd_{t + \Delta t} += \Delta \Udd\f$
\>\>\> \f$ \U_{t + \Delta t} += \beta \Delta t^2 \Delta \Udd \f$
\>\>\> \f$ \Ud_{t + \Delta t} += \gamma \Delta t \Delta \Udd \f$
\>\> \}
\>\> theModel-\f$>\f$setResponse\f$(\U_{t + \Delta t}, \Ud_{t+\Delta t},
\Udd_{t+\Delta t})\f$
\>\> theModel-\f$>\f$setUpdateDomain()
\end{tabbing}
//! Returns \f$0\f$ if successful. A warning message is printed and a negative number
//! returned if an error occurs: \f$-1\f$ if no associated AnalysisModel, \f$-2\f$
//! if the Vector objects have not been created, \f$-3\f$ if the Vector
//! objects and \f$\delta U\f$ are of different sizes.

{\em int sendSelf(int commitTag, Channel \&theChannel); } 
//! Places in a Vector of size 6 the values of \f$\beta\f$, \f$\gamma\f$, {\em
//! dispFlag}, RayleighDampingFlag, \f$\alpha_M\f$ and \f$\beta_K\f$.  Then
//! invokes sendVector() on the Channel with this Vector. Returns
\f$0\f$ if successful, a warning message is printed and a \f$-1\f$ is 
//! returned if \p theChannel fails to send the Vector. 

{\em int recvSelf(int commitTag, Channel \&theChannel, 
//! FEM\_ObjectBroker \&theBroker); } 
//! Receives in a Vector of size 6 the values of \f$\beta\f$, \f$\gamma\f$, {\em
//! dispFlag}, RayleighDampingFlag, \f$\alpha_M\f$ and \f$\beta_K\f$. Returns \f$0\f$
//! if successful. A warning message is printed, \f$\gamma\f$ is set to 0.5,
\f$\beta\f$ to 0.25 and the Rayleigh damping flag set to \p false, and
//! a \f$-1\f$ is returned, if \p theChannel fails to receive the Vector. 

{\em int Print(OPS\_Stream \&s, int flag = 0);}
//! The object sends to \f$s\f$ its type, the current time, \f$\gamma\f$ and
\f$\beta\f$. If Rayleigh damping is specified, the constants \f$\alpha_M\f$ and
\f$\beta_K\f$ are also printed.






