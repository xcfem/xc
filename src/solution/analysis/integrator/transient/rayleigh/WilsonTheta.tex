%File: ~/OOP/analysis/integrator/WilsonTheta.tex
%What: "@(#) WilsonTheta.tex, revA"

\noindent {\bf Files}
\indent \#include \f$<\tilde{ }\f$/analysis/integrator/WilsonTheta.h\f$>\f$

\noindent {\bf Class Declaration}
\indent class WilsonTheta: public TransientIntegrator

\noindent {\bf Class Hierarchy}
\indent MovableObject
\indent\indent Integrator
\indent\indent\indent IncrementalIntegrator
\indent\indent\indent\indent TransientIntegrator
\indent\indent\indent\indent\indent {\bf WilsonTheta}

\noindent {\bf Description} 
\indent WilsonTheta is a subclass of TransientIntegrator which implements
//! the Wilson\f$\Theta\f$ method. In the Wilson\f$\Theta\f$ method, to determine the 
//! velocities, accelerations and displacements at time \f$t + \theta \Delta
//! t\f$, \f$\theta \ge 1.37\f$, for \f$\U_{t+ \theta \Delta t}\f$ 

\[ \R (\U_{t + \theta \Delta t}) = \P(t + \theta \Delta t) -
\F_I(\Udd_{t+ \theta \Delta t}) 
- \F_R(\Ud_{t + \theta \Delta t},\U_{t + \theta \Delta t}) \]

\noindent where we use following functions to relate \f$\Ud_{t + \theta
\Delta t}\f$ and \f$\Udd_{t + \theta \Delta t}\f$ to \f$\U_{t + \theta \Delta
//! t}\f$ and the response quantities at time \f$t\f$:

\[
\dot \U_{t + \theta \Delta t} = \frac{3}{\theta \Delta t} \left(
\U_{t + \theta \Delta t} - \U_t \right)
 - 2 \dot \U_t + \frac{\theta \Delta t}{2} \ddot \U_t 
\]

\[
\ddot \U_{t + \theta \Delta t} = \frac{6}{\theta^2 \Delta t^2}
\left( \U_{t+\theta \Delta t} - \U_t \right)
 - \frac{6}{\theta \Delta t} \dot \U_t -2 \Udd_t
\]

\noindent which  results in the following for determining the responses at
\f$t + \theta \Delta t\f$ 

\[ \left[ \frac{6}{\theta^2 \Delta t^2} \M + \frac{3}{\theta \Delta t}
\C + \K \right] \Delta \U_{t + \theta \Delta t}^{(i)} = \P(t + \theta
\Delta t) - \F_I\left(\Udd_{t+\theta \Delta  t}^{(i-1)}\right) 
- \F_R\left(\Ud_{t + \theta \Delta t}^{(i-1)},\U_{t + \theta \Delta
//! t}^{(i-1)}\right) \]


\noindent The response quantities at time \f$t + \Delta t\f$ are then
//! determined using the following

\[
\Udd_{t + \Delta t} = \Udd_t + \frac{1}{\theta} \left( \Udd_{t +
\theta \Delta t} - \Udd_t \right)
\]

\[ \Ud_{t + \Delta t} = \Ud_t + \frac{\Delta t}{2}\left( \Udd_{t +
\Delta t} + \Udd_t \right) \]

\[ \U_{t + \Delta t} = \U_t + \Delta t\Ud_t + \frac{\Delta t^2}{6}\left(
\Udd_{t + \Delta t} + 2 \Udd_t \right) \]


\pagebreak
\noindent {\bf Class Interface}
\indent // Constructors
\indent {\em WilsonTheta();} 
\indent {\em WilsonTheta(double theta);} 
\indent {\em WilsonTheta(double theta, double alphaM, double betaK);}
\indent // Destructor
\indent {\em virtual~ \f$\tilde{}\f$WilsonTheta();}
\indent // Public Methods
\indent {\em int formEleTangent(FE\_Element *theEle);}
\indent {\em int formNodTangent(DOF\_Group *theDof);}
\indent {\em int domainChanged(void);}
\indent {\em int newStep(double deltaT);}
\indent {\em int update(const Vector \&\f$\Delta U\f$);} 
\indent {\em int commit(void);}
\indent // Public Methods for Output
\indent {\em int sendSelf(int commitTag, Channel \&theChannel);} 
\indent {\em int recvSelf(int commitTag, Channel \&theChannel,
//! FEM\_ObjectBroker \&theBroker);} 
\indent {\em int Print(OPS\_Stream \&s, int flag = 0);}

\noindent {\bf Constructors}
\indent {\em WilsonTheta();} 
//! The integer INTEGRATOR\_TAGS\_WilsonTheta is passed to the TransientIntegrator
//! constructor. \f$\Theta\f$ is set to 0.0. This constructor should only be
//! invoked by an FEM\_ObjectBroker.



\indent {\em WilsonTheta(double theta);} 
//! Sets \f$\Theta\f$ to \p theta, \f$\gamma\f$ to \f$(1.5 - \alpha)\f$ and \f$\beta\f$
//! to \f$0.25*\alpha^2\f$. addition, a flag is set indicating that Rayleigh
//! damping will not be used. 

\indent {\em WilsonTheta(double theta);} 
//! Sets \f$\Theta\f$ to \p theta, \f$\gamma\f$ to \f$(1.5 - \alpha)\f$ and \f$\beta\f$
//! to \f$0.25*\alpha^2\f$. In addition, a flag is set indicating that Rayleigh
//! damping will not be used.

\indent {\em WilsonTheta(double theta, double alphaM, double betaK);} 
//! This constructor is invoked if Rayleigh damping is to be used, 
//! i.e. \f$\D = \alpha_M M + \beta_K K\f$. Sets \f$\Theta\f$ to \p theta,
\f$\gamma\f$ to \f$(1.5 - \alpha)\f$, \f$\beta\f$ to \f$0.25*\alpha^2\f$, \f$\alpha_M\f$ to
\p alphaM and \f$\beta_K\f$ to \p betaK. Sets a flag indicating whether the
//! incremental solution is done in terms of displacement or acceleration
//! to \p dispFlag and a flag indicating that Rayleigh damping will 
//! be used. 


\noindent {\bf Destructor}
\indent {\em virtual~ \f$\tilde{}\f$WilsonTheta();} 
//! Invokes the destructor on the Vector objects created.

\noindent {\bf Public Methods}
\indent {\em int formEleTangent(FE\_Element *theEle);}
//! This tangent for each FE\_Element is defined to be \f$\K_e = c1 \K
+ c2  \D + c3 \M\f$, where c1,c2 and c3 were determined in the last invocation
//! of the newStep() method. Returns \f$0\f$ after performing the
//! following operations:  
\begin{tabbing}
//! while \= \+ while \= while \= \kill
//! if (RayleighDamping == false) \{ \+
//! theEle-\f$>\f$zeroTang()
//! theEle-\f$>\f$addKtoTang(c1)
//! theEle-\f$>\f$addCtoTang(c2)
//! theEle-\f$>\f$addMtoTang(c3)  \-
\} else \{ \+
//! theEle-\f$>\f$zeroTang()
//! theEle-\f$>\f$addKtoTang(c1 + c2 * \f$\beta_K\f$)
//! theEle-\f$>\f$addMtoTang(c3 + c2 * \f$\alpha_M\f$)  \- 
\}
\end{tabbing}

{\em int formNodTangent(DOF\_Group *theDof);}
//! This performs the following:
\begin{tabbing}
//! while \= \+ while \= while \= \kill
//! theDof-\f$>\f$zeroUnbalance()
//! if (RayleighDamping == false)  \+
//! theDof-\f$>\f$addMtoTang(c3)  \-
//! else \+
//! theDof-\f$>\f$addMtoTang(c3 + c2 * \f$\alpha_M\f$)  \- 
\end{tabbing}


{\em int domainChanged(void);}
//! If the size of the LinearSOE has changed, the object deletes any old Vectors
//! created and then creates \f$6\f$ new Vector objects of size equal to {\em
//! theLinearSOE-\f$>\f$getNumEqn()}. There is a Vector object created to store
//! the current displacement, velocity and accelerations at times \f$t\f$ and
\f$t + \Delta t\f$ (between newStep()} and {\em commit() the \f$t +
\Delta t\f$ quantities store \f$t + \Theta \Delta t\f$ quantities).
//! The response quantities at time \f$t + \Delta t\f$ are
//! then set by iterating over the DOF\_Group objects in the model and
//! obtaining their committed values. 
//! Returns \f$0\f$ if successful, otherwise a warning message and a negative
//! number is returned: \f$-1\f$ if no memory was available for constructing
//! the Vectors.


{\em int newStep(double \f$\Delta t\f$);}
//! The following are performed when this method is invoked:
\begin{enumerate}
\item First sets the values of the three constants {\em c1}, {\em c2}
//! and {\em c3}: {\em c1} is set to \f$1.0\f$, 
{\em c2} to \f$3 / (\Theta
\Delta t)\f$ and {\em c3} to \f$6 / (\Theta \Delta t)^2)\f$. 
\item Then the Vectors for response quantities at time \f$t\f$ are set
//! equal to those at time \f$t + \Delta t\f$.
\begin{tabbing}
//! while \= while \= while \= while \= \kill
\>\> \f$ \U_t = \U_{t + \Delta t}\f$
\>\> \f$ \Ud_t = \Ud_{t + \Delta t} \f$
\>\> \f$ \Udd_t = \Udd_{t + \Delta t} \f$ 
\end{tabbing}
\item Then the velocity and accelerations approximations
//! at time \f$t + \Theta \Delta t\f$ are set using the difference
//! approximations,
\begin{tabbing}
//! while \= while \= while \= while \= \kill
\>\> \f$ \U_{t + \theta \Delta t} = \U_t \f$
\>\> \f$ \dot \U_{t + \theta \Delta t} = - 2 \dot \U_t + \frac{\theta
\Delta t}{2} \ddot \U_t  \f$
\>\> \f$ \ddot \U_{t + \theta \Delta t} = - \frac{6}{\theta \Delta t}
\dot \U_t -2 \Udd_t \f$ 
\end{tabbing}
\item The response quantities at the DOF\_Group objects are updated
//! with the new approximations by invoking setResponse() on the
//! AnalysisModel with quantities at time \f$t + \Theta \Delta t\f$.
\begin{tabbing}
//! while \= while \= while \= while \= \kill
\>\> theModel-\f$>\f$setResponse\f$(\U_{t + \theta \Delta t}, \Ud_{t+\theta
\Delta t}, \Udd_{t+ \theta \Delta t})\f$ 
\end{tabbing}
\item current time is obtained from the AnalysisModel, incremented by
\f$\Theta \Delta t\f$, and {\em applyLoad(time, 1.0)} is invoked on the
//! AnalysisModel. 
\item Finally updateDomain() is invoked on the AnalysisModel.
\end{enumerate}
//! The method returns \f$0\f$ if successful, otherwise a negative number is
//! returned: \f$-1\f$ if \f$\gamma\f$ or \f$\beta\f$ are \f$0\f$, \f$-2\f$ if \p dispFlag
//! was true and \f$\Delta t\f$ is \f$0\f$, and \f$-3\f$ if domainChanged()
//! failed or has not been called.



{\em int update(const Vector \&\f$\Delta U\f$);}
//! Invoked this first causes the object to increment the DOF\_Group
//! response quantities at time \f$t + \Theta \Delta t\f$. The displacement Vector is  
//! incremented by \f$ c1 * \Delta U\f$, the velocity Vector by \f$
//! c2 * \Delta U\f$, and the acceleration Vector by \f$c3 * \Delta U\f$. 
//! The response quantities at the DOF\_Group objects are then updated
//! with the new approximations by invoking setResponse() on the
//! AnalysisModel with displacements, velocities and accelerations at time
\f$t + \Theta \Delta t\f$.
//! Finally updateDomain() is invoked on the AnalysisModel. 
\begin{tabbing}
//! while \= while \= while \= while \= \kill
\>\> \f$ \U_{t + \theta \Delta t} += \Delta \U\f$
\>\> \f$ \dot \U_{t + \theta \Delta t} += \frac{3}{\theta \Delta t}
\Delta \U  \f$
\>\> \f$ \ddot \U_{t + \theta \Delta t} += \frac{6}{\theta^2 \Delta
//! t^2} \Delta \U \f$ 
\>\> theModel-\f$>\f$setResponse\f$(\U_{t + \alpha \theta t}, \Ud_{t+\theta
\Delta t}, \Udd_{t+ \theta \Delta t})\f$
\>\> theModel-\f$>\f$updateDomain()
\end{tabbing}
//! Returns \f$0\f$ if successful. A warning message is printed and a negative number
//! returned if an error occurs: \f$-1\f$ if no associated AnalysisModel, \f$-2\f$
//! if the Vector objects have not been created, \f$-3\f$ if the Vector
//! objects and \f$\Delta U\f$ are of different sizes.



{\em int commit(void);}
//! First the quantities at time \f$t + \Delta t\f$ are determined using the
//! quantities at time \f$t\f$ and \f$t + \Theta \Delta t\f$.
//! Then the response quantities at the DOF\_Group objects are updated
//! with the new approximations by invoking setResponse() on the
//! AnalysisModel with displacement, velocity and accelerations at time \f$t +
\Delta t\f$. The time is obtained from the AnalysisModel and \f$(\Theta
-1) \Delta t\f$ is subtracted from the value. The time is set in the
//! Domain by invoking {\em setCurrentDomainTime(time)} on the
//! AnalysisModel. Finally updateDomain()} and {\em commitDomain()
//! are invoked on the AnalysisModel. 
\begin{tabbing}
//! while \= while \= while \= while \= \kill
\>\> \f$\Udd_{t + \Delta t} = \Udd_t + \frac{1}{\theta} \left( \Udd_{t +
\theta \Delta t} - \Udd_t \right)\f$
\>\> \f$ \Ud_{t + \Delta t} = \Ud_t + \frac{\Delta t}{2}\left( \Udd_{t +
\Delta t} + \Udd_t \right) \f$
\>\> \f$ \U_{t + \Delta t} = \U_t + \Delta t\Ud_t + \frac{\Delta t^2}{6}\left(
\Udd_{t + \Delta t} + 2 \Udd_t \right) \f$
\>\> theModel-\f$>\f$setResponse\f$(\U_{t + \Delta t}, \Ud_{t+
\Delta t}, \Udd_{t+\Delta t})\f$
\>\> time = theModel-\f$>\f$getDomainTime()
\>\> time -= \f$(\theta -1) * \Delta t\f$
\>\> theModel-\f$>\f$setTime(time)
\>\> theModel-\f$>\f$commitDomain()
\end{tabbing}
//! Returns \f$0\f$ if successful, a warning
//! message and a negative number if not: \f$-1\f$ if no AnalysisModel
//! associated with the object and \f$-2\f$ if commitDomain() failed.


{\em int sendSelf(int commitTag, Channel \&theChannel); } 
//! Places \f$\Theta\f$, rayleigh damping flag, \f$\alpha_M\f$ and \f$\beta_K\f$ in a
//! vector if size 4 and invokes \p sendVector on the Channel with this
//! Vector. Returns \f$0\f$ if successful, a warning message is printed and a
\f$-1\f$ is returned if \p theChannel fails to send the Vector. 

{\em int recvSelf(int commitTag, Channel \&theChannel, 
//! FEM\_ObjectBroker \&theBroker); } 
//! Receives in a Vector of size 4 the value of \f$\Theta\f$, the rayleigh
//! damping flag, \f$\alpha_M\f$ and \f$\beta_K\f$.. Returns \f$0\f$ if
//! successful, a warning message is printed,  \f$\Theta\f$ is set to \f$0\f$, the
//! rayleigh damping flag to \p false, and a \f$-1\f$ is returned if {\em
//! theChannel} fails to receive the Vector. 

{\em int Print(OPS\_Stream \&s, int flag = 0);}
//! The object sends to \f$s\f$ its type, the current time, \f$\alpha\f$, \f$\gamma\f$ and
\f$\beta\f$. 

