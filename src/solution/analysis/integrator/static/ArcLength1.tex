%File: ~/OOP/analysis/integrator/ArcLength1.tex
%What: "@(#) ArcLength1.tex, revA"

\noindent {\bf Files}
\indent \#include \f$<\tilde{ }\f$/analysis/integrator/ArcLength1.h\f$>\f$

\noindent {\bf Class Declaration}
\indent class ArcLength1: public StaticIntegrator

\noindent {\bf Class Hierarchy}
\indent MovableObject
\indent\indent Integrator
\indent\indent\indent IncrementalIntegrator
\indent\indent\indent\indent StaticIntegrator
\indent\indent\indent\indent\indent {\bf ArcLength1}

\noindent {\bf Description} 
\indent ArcLength1 is a subclass of StaticIntegrator, it is
//! used to when performing a static analysis on the FE\_Model using a
//! simplified form of the arc length method. In the arc length method
//! implemented by this class, the following constraint equation is added to
//! equation~\ref{staticFormTaylor} of the StaticIntegrator class: 

\begin{equation}
\Delta \U_n^T \Delta \U_n  + \alpha^2 \Delta \lambda_n^2  = \Delta s^2
\end{equation}

//! where 

\[
\Delta \U_n = \sum_{j=1}^{i} \Delta \U_n^{(j)} = \Delta \U_n^{(i)} +
//! d\U^{(i)} 
\]

\[
\Delta \lambda_n = \sum_{j=1}^{i} \Delta \lambda_n^{(j)} = \Delta \lambda_n^{(i)} +
//! d\lambda^{(i)} 
\]

\noindent this equation cannot be added directly into
//! equation~\ref{staticFormTaylor} to produce a linear system of \f$N+1\f$
//! unknowns. To add this equation we make some assumptions ala Yang
(REF), which in so doing allows us to solve a system of \f$N\f$
//! unknowns using the method of ??(REF).  Rewriting
//! equation~\ref{staticFormTaylor} as  

\[
\K_n^{(i)} \Delta \U_n^{(i)} = \Delta \lambda_n^{(i)} \P +
\lambda_n^{(i)} \P - \F_R(\U_n^{(i)}) = \Delta \lambda_n^{(i)} \P + \R_n^{(i)}
\]

\noindent The idea of ?? is to separate this into two equations:

\def\Uh{\dot{\bf U}}
\def\Ub{\overline{\bf U}}

\[
\K_n^{(i)} \Delta \Uh_n^{(i)} = \P
\]

\[
\K_n^{(i)} \Delta \Ub_n^{(i)} = \R_n^{(i)}
\]

\noindent where now

\begin{equation}
 \Delta \U_n^{(i)} = \Delta \lambda_n^{(i)} \Delta \Uh_n^{(i)} +
\Delta \Ub_n^{(i)}  
\label{splitForm}
\end{equation}

\noindent We now rewrite the constraint equation based on two conditions:

\begin{enumerate}
\item {\bf \f$i = 1\f$}: assuming \f$\R_n^{(1)} = \zero\f$, i.e. the system is
//! in equilibrium at the start of the iteration, the following is obtained

\[  \Delta \U_n^{(1)} = \Delta \lambda_n^{(1)} \Delta \Uh_n^{(1)} + \zero \]

\[ \Delta \lambda_n^{(1)} = \begin{array}{c} + - \end{array}
\sqrt{\frac{\Delta s^2}{\Uh^T \Uh + \alpha^2}} \]

\noindent The question now is whether {\bf +} or {\bf -} should be
//! used. In this class, \f$d \lambda\f$ from the previous iteration \f$(n-1)\f$
//! is used, if it was positive {\bf +} is assumed, otherwise {\bf -}. This may
//! change. There are other ideas: ?(REF) number of negatives on diagonal
//! of decomposed matrix, ...

\item {\bf \f$i > 1\f$}

\[ \left( \Delta \U_n^{(i)} + d\U^{(i)} \right)^T \left( \Delta \U_n^{(i)} +
//! d\U^{(i)} \right) + \alpha^2 \left( \Delta \lambda_n^{(i)} + d\lambda^{(i)}
\right)^2 = \Delta s^2 \]

\[
\Delta {\U_n^{(i)}}^T\Delta \U_n^{(i)} + 2{d\U^{(i)}}^T\Delta \U_n^{(i)} + {d\U^{(i)}}^T d\U^{(i)}
+ \alpha^2 \Delta {\lambda_n^{(i)}}^2
+ 2 \alpha^2 d\lambda^{(i)} \Delta \lambda_n^{(i)} + \alpha^2 {d\lambda^{(i)}}^2
= \Delta s^2
\]

\noindent assuming the constraint equation was solved at \f$i-1\f$,
//! i.e. \f${d\U^{(i)}}^T d\U^{(i)} + \alpha^2 {d\lambda^{(i)}}^2 = \Delta s^2\f$, this reduces to

\[
\Delta {\U_n^{(i)}}^T\Delta \U_n^{(i)} + 2{d\U^{(i)}}^T\Delta \U_n^{(i)} + 
\alpha^2 \Delta {\lambda_n^{(i)}}^2
+ 2 \alpha^2 d\lambda^{(i)} \Delta \lambda_n^{(i)} 
= 0
\]

 \noindent For our ArcLength1 method we make the ADDITIONAL assumption that
\f$ 2{d\U^{(i)}}^T\Delta \U_n^{(i)} 
+ 2 \alpha^2 d\lambda^{(i)} \Delta \lambda_n^{(i)} \f$ \f$>>\f$
\f$ 
\Delta {\U_n^{(i)}}^T\Delta \U_n^{(i)} +
\alpha^2 \Delta {\lambda_n^{(i)}}^2
\f$
//! the constraint equation at step \f$i\f$ reduces to

\[
{d\U^{(i)}}^T\Delta \U_n^{(i)} 
+ \alpha^2 d\lambda^{(i)} \Delta \lambda_n^{(i)} = 0
\]

\noindent hence if the class was to solve an \f$N+1\f$ system of equations at
//! each step, the system would be:

\[ \left[
\begin{array}{cc}
\K_n^{(i)} & -\P
{d\U^{(i)}}^T & \alpha^2 d\lambda^{(i)} 
\end{array} \right] 
\left\{
\begin{array}{c}
\Delta \U_n^{(i)}
\Delta \lambda_n^{(i)}
\end{array} \right\} = \left\{
\begin{array}{c}
\lambda_n^{(i)} \P - \F_R(\U_n^{(i)})
0
\end{array} \right\}
\]

\noindent instead of solving an \f$N+1\f$ system, equation~\ref{splitForm}
//! is used to give

\[
{d\U^{(i)}}^T \left(\Delta \lambda_n^{(i)} \Delta \Uh_n^{(i)} + \Delta
\Ub_n^{(i)}\right) 
+ \alpha^2 d\lambda^{(i)} \Delta \lambda_n^{(i)} = 0
\]

\noindent which knowing \f$\Uh_n^{(i)}\f$ and \f$\Ub_n^{(i)}\f$ can
//! be solved for \f$\Delta \lambda_n^{(i)}\f$ 

\[
\Delta \lambda_n^{(i)} = -\frac{{d\U^{(i)}}^T \Delta \Ub_n^{(i)}}{{d\U^{(i)}}^T \Delta
\Uh_n^{(i)} + \alpha^2 d\lambda^{(i)}}
\]

\end{enumerate}
\noindent {\bf Class Interface}
\indent // Constructors
\indent {\em ArcLength1(double arc, double \f$\alpha\f$ = 1.0);}
\indent // Destructor
\indent {\em \f$\tilde{ }\f$ArcLength1();}
\indent // Public Methods
\indent {\em int newStep(void);}
\indent {\em int update(const Vector \&\f$\Delta U\f$);}
\indent {\em int domainChanged(void); }
\indent // Public Methods for Output
\indent {\em int sendSelf(int commitTag, Channel \&theChannel);} 
\indent {\em int recvSelf(int commitTag, Channel \&theChannel,
//! FEM\_ObjectBroker \&theBroker);} 
\indent {\em int Print(OPS\_Stream \&s, int flag = 0);}

\noindent {\bf Constructors}
\indent {\em ArcLength1(double dS, double alpha = 1.0);}
//! The integer INTEGRATOR\_TAGS\_ArcLength1 (defined in
\f$<\f$classTags.h\f$>\f$) is passed to the StaticIntegrator classes
//! constructor. The value of \f$\alpha\f$ is set to \p alpha and 
\f$\Delta s\f$ to \p dS.

\noindent {\bf Destructor}
\indent {\em \f$\tilde{ }\f$ArcLength1();} 
//! Invokes the destructor on the Vector objects created in {\em
//! domainChanged()}.

\noindent {\bf Public Methods}

{\em int newStep(void);}
//! newStep() performs the first iteration, that is it solves for 
\f$\lambda_n^{(1)}\f$ and \f$\Delta \U_n^{(1)}\f$ and updates the model with
\f$\Delta \U_n^{(1)}\f$ and increments the load factor by
\f$\lambda_n^{(1)}\f$. To do this it must set the rhs of the LinearSOE to
\f$\P\f$, invoke formTangent() on itself and solve the LinearSOE to
//! get \f$\Delta \Uh_n^{(1)}\f$.

{\em int update(const Vector \&\f$\Delta U\f$);}
//! Note the argument \f$\Delta U\f$ should be equal to \f$\Delta \Ub_n^{(i)}\f$.
//! The object then determines \f$\Delta \Uh_n^{(i)}\f$ by setting the rhs of
//! the linear system of equations to be \f$\P\f$ and then solving the
//! linearSOE. It then solves for
\f$\Delta \lambda_n^{(i)}\f$ and \f$\Delta \U_n^{(i)}\f$ and updates the model with
\f$\Delta \U_n^{(i)}\f$ and increments the load factor by \f$\Delta
\lambda_n^{(i)}\f$. Sets the vector \f$x\f$ in the LinearSOE object to be
//! equal to \f$\Delta \U_n^{(i)}\f$ before returning (this is for the
//! convergence test stuff.


\indent {\em int domainChanged(void); } 
//! The object creates the Vector objects it needs. Vectors are created to
//! stor \f$\P\f$, \f$\Delta \Ub_n^{(i)}\f$, \f$\Delta \Uh_n^{(i)}\f$, \f$\Delta
\Ub_n^{(i)}\f$, \f$dU^{(i)}\f$. To form \f$\P\f$, the current load factor is
//! obtained from the model, it is incremented by \f$1.0\f$, {\em
//! formUnbalance()} is invoked on the object, and the \f$b\f$ vector is
//! obtained from the linearSOE. This is \f$\P\f$, the load factor on the
//! model is then decremented by \f$1.0\f$.

{\em int sendSelf(int commitTag, Channel \&theChannel); } 
//! Places the values of \f$\Delta s\f$ and \f$\alpha\f$ in a
//! vector of size \f$2\f$ and invokes sendVector() on \p theChannel.
//! Returns \f$0\f$ if successful, a warning message is printed and
//! a \f$-1\f$ is returned if \p theChannel fails to send the Vector.

{\em int recvSelf(int commitTag, Channel \&theChannel, 
//! FEM\_ObjectBroker \&theBroker); } 
//! Receives in a Vector of size 2 the values of \f$\Delta s\f$ and \f$\alpha\f$.
//! Returns \f$0\f$ if successful, a warning message is printed, \f$\delta
\lambda\f$ is set to \f$0\f$, and a \f$-1\f$ is returned if \p theChannel
//! fails to receive the Vector.

{\em int Print(OPS\_Stream \&s, int flag = 0);}
//! The object sends to \f$s\f$ its type, the current value of \f$\lambda\f$, and
\f$\delta \lambda\f$. 
