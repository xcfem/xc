\section{Introduction}
This Appendix has the object of defining the actions, weighting coefficients and the combination of actions which shall be taken into account when designing structures.

Checking the structures through design is the most used method to guarantee their safety \footnote{Other procedures are also acceptable such as the reduced model tests, full-scale tests of the structure or its elements, extrapolation of the behaviour of similar structures, \ldots}.

\subsection{The Limit States design method} \label{limit_states_introd}
The usual method prescribed by the codes for checking the safety of a structure is the so-called \emph{Method of limit states}. A \emph{limit state} is a situation in which, when exceeded, it may be considered that the structure does not fulfil one of the functions for which it has been designed.

The limit states are classified in:
\begin{itemize}
\item \emph{Ultimate Limit States (ULS)};
\item \emph{Serviceability Limit States (SLS)}, and
\item \emph{Durability Limit States (DLS)}.
\end{itemize}

\subsection{Design situations}
The concept of \emph{design situation} is useful to sort the checks performed on the project or study of a structure. A design situation is a simplified representation of the reality that is amenable to analysis.

Thus, it can be considered design situations those that correspond to the different phases of construction of the structure, the normal use of the structure, its reparation, exceptional conditions, \ldots. 

For each of the design situations, it must be checked that the structure doesn't exceed any of the Limit States previously laid down in paragraph \ref{limit_states_introd}

\subsection{Actions}
\emph{Action} is defined as any cause capable of producing stress states in a structure, or modifying the existing one. Weight coefficients can be different according to the codes that apply for verification of the different structural elements (IAP, EHE, Eurocodes,\ldots).

\subsection{Working life} \label{sc_vida_util}
The working life of a structure is the period of time from the end of its execution, during which must maintain the requirements of security and functionality of project and an acceptable aesthetic appearance. During that period it will require conservation in accordance with the maintenance plan established for that purpose.

The design working life depends on the type of structure and must be fixed by the Owners at the start of the design. In any case its duration will be lower than that indicated in the regulations applicable or, in the absence of these, than the values laid down in Table \ref{tb_vid_util}.


\begin{table}
\begin{center}
\begin{small}
\begin{tabular}{|p{6cm}|l|}
\hline
\textbf{Type of structure} & \textbf{Design working life} \\
\hline
Temporary structures (*) & 3 to 10 years (*) \\
\hline
Replaceable structural elements that are not part of the main structure (eg, handrails, pipe supports) & 10 to 25 years \\
\hline
Agricultural or industrial buildings (or installations) and maritime works & 15 to 50 years\\
\hline
Residential buildings or offices, bridges or crossings of a total length of less than 10 meters and civil engineering structures (except maritime works) having a low or average economic impact & 50 years \\
\hline
Public buildings, health and education. & 75 years\\
\hline
Monumental buildings or having a special importance & 100 years \\
\hline
Bridges of total length equal to or greater than 10 meters and other civil engineering structures of high economic impact & 100 years \\
\hline
\multicolumn{2}{|p{9.5cm}|}{(*)In accordance with the purpose of the structure (temporary exposure, etc.). Under no circumstances shall structures with a design working life greater than 10 years be regarded as temporary structures.} \\
\hline
\end{tabular}
\caption{Design working life of the various types of structure (according reference \cite{EAE}).} \label{tb_vid_util}
\end{small}
\end{center}
\end{table}
When a structure consists of different members, different working life values may be adopted for such members, always in accordance with the type and characteristics of the construction thereof.

\subsection{Risk level}
The level of risk of an infrastructure defines the consequences of a structural failure during its construction or service (public building, private store, bridge, \ldots)

\subsection{Control level}
Regardless of the rigor with which the checking calculations of the structure are made during the project, its safety also depend on careful construction of it. Different standards establish the influence that the level of control during the execution of the work has on safety factors to be used in the execution of the same.

\subsection{Combination of actions}
When designing a structure or a structural member by the limit state method, load combinations shall be considered as the sum of the products of the load effect corresponding to the basic value of each load and the load factor.

Load factors shall be determined appropriately considering the limit state, the target reliability index, the variability in the load effect of each load and resistance, the probability of load coincidence, etc.

\subsection{Verification of the structure}
From the discussion in the previous sections, the verification procedure of the structure will consist of performing the following tasks:

\begin{enumerate}
\item identify the design situations to be considered when checking the structure;
\item identify the load criterions hypotheses for each of those design situations;
\item define the combinations of actions to be considered when checking the ULS and SLS, depending on:

\begin{enumerate}
\item materials composing the structure or the element to check: rolled steel, reinforced concrete, wood, \ldots;
\item risk level of the infrastructure
\item level of control with which the construction work is performed;
\item design situation (persistent, transient or accidental)
\end{enumerate}
\item obtain the calculation value of the effect of actions for each combination.
\item verify all the limit states.
\end{enumerate}

\section{Actions} \label{sc_acciones}
An action is a set of forces applied to the structure  or a set of imposed deformations or accelerations, that has an effect on structural members (e.g. internal force, moment, stress, strain) or on the whole structure (e.g. defection, rotation)

\subsection{Classification of actions}
Actions can be classified by their variation over time, their nature, their origin, their spatial variation, \ldots

\subsubsection{By their nature}
\begin{itemize}
\item \textbf{Direct actions}: loads applied to the structure (e.g. self-weight, dead load, live load, \ldots)
\item \textbf{Indirect actions}: imposed deformations or accelerations caused for example by temperature changes, moisture variation,\ldots
\end{itemize}

\subsubsection{By their variation over time} \label{sc_var_tiempo}
Actions shall be classified by their variation in time, by reference to their \emph{service life}\footnote{See section \ref{sc_vida_util}.}, as follows:

\begin{itemize}
\item \textbf{Permanent actions G}: actions that are likely to act throughout a given reference period and for which the variation in magnitude with time is negligible, or for which the variation is always in the same direction (monotonic) until the action attains a certain limit value, e.g. self-weight of structures, fixed equipment and road surfacing, and indirect actions caused by shrinkage and uneven settlement.
\item \textbf{Permanents of a non-constant value G*}: are those which act at any time but whose magnitude is non constant. This group include those actions whose variation is a function of elapsed time and are produced in a single direction, tending towards a certain limit value (rheological actions, pretensioning, subsidence of the ground under the foundations, \ldots). They also include other actions originating from the ground whose magnitude does not vary as a function of time but as a function of the interaction between the ground and the structure (for example, thrusts on vertical elements).
\item \textbf{Variables Q}: action for which the variation in magnitude with time is neither negligible nor monotonic. E.g. imposed loads on building floors, beams and roofs, wind actions or snow loads.  

\item \textbf{Accidental actions A}: action, usually of short duration but of significant magnitude, that is unlikely to occur
on a given structure during the design working life. E.g. explosions, or impact from vehicles.

\item \textbf{seismic action AS}: action that arises due to earthquake ground motions.

\end{itemize}

\subsubsection{By their origin}
\begin{itemize}
\item \textbf{Gravitational}: which has its origin in the earth's gravitational field (self-weight, dead load, earth pressure, \ldots)
\item \textbf{Climatic}: whose origin is in the climate (thermal action and wind actions\footnote{thermal and wind actions can not be due to climate, such as in the case of an oven or structures subjected to the thrust of jet engines of aircraft})
\item \textbf{Rheological}: which has its origin in the response of material with plastic flow rather than deforming elastically when a force is applied (e.g. shrinkage of concrete).
\item \textbf{Seismic}: due to earthquake ground motions.
\end{itemize}






\subsubsection{By the structural response which they produce} 
\begin{itemize}
\item \textbf{static action}: action that does not cause significant acceleration of the structure or structural members;
\item \textbf{dynamic action}:  action that causes significant acceleration of the structure or structural members;
\item \textbf{quasi-static action}: dynamic action represented by an equivalent static action in a static model.
\end{itemize}


\subsubsection{By their spatial variation}
\begin{itemize}
\item \textbf{fixed action}: action that has a fixed distribution and position over the structure or structural member such that the magnitude and direction of the action are determined unambiguously for the whole structure or structural member if this magnitude and direction are determined at one point on the structure or structural member;
\item \textbf{free action}: action that may have various spatial distributions over the structure.
\end{itemize}

\subsubsection{By their relation with other actions} \label{sc_acc_rel_otras}
\begin{itemize}
\item \textbf{Compatible actions}: two actions are compatible when it's possible for them to act simultaneously.
\item \textbf{Incompatible actions}: two actions are incompatible when it's impossible for them to act at the same time (e.g. one crane acting simultaneously in two different positions).
\item \textbf{Synchronous actions}: two actions are synchronous when the act necessarily together, at the same time (e.g. the braking load of a crane bridge will be synchronised with the action of the weight of the crane).
\end{itemize}

\subsubsection{By their participation in a combination} \label{sc_modo_partic_acc}
\begin{itemize}
\item \textbf{Leading action}: in a combination of actions, the leading variable action is the one which produces the largest design load effect; its characteristic value is used.
\item \textbf{Accompanying action}: variable action that accompanies the leading action in a combination; its characteristic value is reduced by using a factor $\Psi$.
\end{itemize}

\subsection{Values of actions} \label{sc_val_acciones}

\subsubsection{Characteristic value of an action $F_k$} \label{sc_val_carac}
It is the principal representative value of an action; it is chosen so as to correspond to a 5\% probability of not being exceeded on the unfavourable side during a "reference period" taking into account the design working life of the structure and the duration of the design situation.

\subsubsection{Combination value of a variable action $F_{r0}$}
Value chosen so that the probability that the effects caused by the combination will be exceeded is approximately the same as by the characteristic value of an individual action. It may be expressed as a determined part of the characteristic value by using a factor $\Psi_0 \le 1$

\subsubsection{Frequent value of a variable action $F_{r1}$}
Value determined so that either the total time, within the reference period, during which it is exceeded is only a small given part
of the reference period, or the frequency of it being exceeded is limited to a given value. It may be expressed as a determined part of the characteristic value by using a factor $\Psi_1 \le 1$.

\subsubsection{Quasi-permanent value of a variable action $F_{r2}$}
Value determined so that the total period of time for which it will be exceeded is a large fraction \footnote{according to \emph{Documento Nacional de Aplicaci\'{o}n espa\~{n}ol del Euroc\'{o}digo de Hormig\'{o}n (UNE ENV 1992-1-1)} more than half of the service life of the structure} of the reference period. It may be expressed as a determined part of the characteristic value by using a factor $\Psi_1 \le 2$.

\subsubsection{Representative value $F_r$ of the actions. Factors of simultaneity} \label{sc_coef_simult}
The representative value of an action is the value of it that is used to verify the limit states. By multiplying this representative value by the the corresponding partial coefficient $\gamma_f$, the calculation value shall be obtained.

The principal representative value of the actions is their characteristic value. Usually, for permanent and accidental actions, a single representative value is considered, that matches the characteristic value ($\psi= 1$) \footnote{The IAP instruction  (reference \cite{IAP}) makes some exceptions to this rule)}.
Other representative values are considered for the variable actions, in accordance with the verification involved and the type of action:

\begin{itemize}
\item \textbf{Characteristic value $F_k$}: this value is used for leading actions in the verification of ultimate limit states in a continuous or temporary situation and of irreversible serviceability limit states.
\item \textbf{Combination value $\psi= \psi_{0}F_k$} this value is used for accompanying actions in the verification of ultimate limit states in a continuous or temporary situation and of irreversible serviceability limit states.
\item \textbf{Frequent value $\psi= \psi_{1}F_k$}: this value is used for the leading action in the verification of ultimate limit states in an accidental situations and of reversible serviceability limit states.
\item \textbf{Quasi-permanent value $\psi= \psi_{2}F_k$}: this value is used for accompanying actions in the verification of ultimate limit states in an accidental situation and of reversible serviceability limit states as well as in the assessment of the postponed effects.
\end{itemize}

In short, the representative value of an action depends on:
\begin{itemize}
\item its variation over time (G,G*,Q,A,AS);
\item its participation in the combination as \emph{leading action} or \emph{accompanying action};
\item the type of situation (accidental, \ldots);
\item the origin of the load (climate, use, water, \ldots).
\end{itemize}

\paragraph{Values of $\Psi$ factors of simultaneity}
The value of the simultaneity factors $\psi$ are different depending on the action that is involved.

\subparagraph{According to EHE:} the recommended values of factors of simultaneity  $\psi_{0}$,$\psi_{1}$,$\psi_{2}$ according to the \emph{Documento Nacional de Aplicaci\'{o}n espa\~{n}ol del Euroc\'{o}digo de Hormig\'{o}n} (UNE ENV 1992-1-1) can be seen in tables \ref{tb_coefs_psi_1EHE} y \ref{tb_coefs_psi_2EHE}.

\begin{table}
\begin{center}
\begin{small}
\begin{tabular}{|l|c|c|c|}
\hline
\textsc{Climatic actions} & $\psi_{0}$ & $\psi_{1}$ & $\psi_{2}$ \\
\hline
Snow loads & 0.6 & 0.2 & 0.0 \\
Wind loads & 0.6 & 0.5 & 0.0 \\
Temperature (\emph{non-fire}) & 0.6 & 0.5 & 0.0 \\
\hline
\end{tabular}
\end{small}
\caption{Recommended values of $\Psi$ factor for climatic actions, according to EHE} \label{tb_coefs_psi_1EHE}
\end{center}
\end{table}

\begin{table}
\begin{center}
\begin{small}
\begin{tabular}{|l|c|c|c|}
\hline
\textsc{Live loads} & $\psi_{0}$ & $\psi_{1}$ & $\psi_{2}$ \\
\hline
\textbf{Roofs} & & & \\
\hline
Inaccessible or accessible only for maintenance & 0.7 & 0.5 & 0.3 \\
Accessible & by use & by use & by use \\
\hline
\textbf{Residential buildings} & & & \\
\hline
Rooms & 0.7 & 0.5 & 0.3 \\
Stairs and public accesses & 0.7 & 0.5 & 0.3 \\
Cantilevered balconies & 0.7 & 0.5 & 0.3 \\
\hline
\textbf{Hotels, hospitals, prisons, \ldots} & & & \\
\hline
Bedrooms & 0.7 & 0.5 & 0.3 \\
Public areas, stairs and accesses & 0.7 & 0.7 & 0.6 \\
Assembly and areas  & 0.7 & 0.7 & 0.6 \\
Cantilevered balconies & by use & by use & by use \\
\hline
\textbf{Office and commercial buildings} & & & \\
\hline
Private premises & 0.7 & 0.5 & 0.3 \\
Public offices & 0.7 & 0.5 & 0.3 \\
Shops & 0.7 & 0.7 & 0.6 \\
Commercial galleries, stairs and access & 0.7 & 0.7 & 0.6 \\
Storerooms & 1.0 & 0.9 & 0.8 \\
Cantilevered balconies & by use & by use & by use \\
\hline
\textbf{Educational buildings} & & & \\
\hline
Classrooms, offices and canteens & 0.7 & 0.7 & 0.6 \\
Stairs and access & 0.7 & 0.5 & 0.6 \\
Cantilevered balconies & by use & by use & by use \\
\hline
\textbf{Churches, buildings for assembly and public performances} & & & \\
\hline
Halls with fixed seatings & 0.7 & 0.7 & 0.6 \\
Halls without fixed seatings, tribunes, stairs & 0.7 & 0.7 & 0.6 \\
Cantilevered balconies & by use & by use & by use \\
\hline
\textbf{Driveways and garages} & & & \\
\hline
Traffic areas with vehicles under 30 kN in weight & 0.7 & 0.7 & 0.6 \\
Traffic areas with vehicles of 30 to 160 kN in weight & 0.7 & 0.5 & 0.3 \\
\hline
\end{tabular}
\end{small}
\end{center}
\caption{Recommended values of $\Psi$ factors of simultaneity for climatic loads, according to EHE} \label{tb_coefs_psi_2EHE}
\end{table}

\subparagraph{According to EAE \cite{EAE} :} see tables \ref{tb_coefs_psi_1EAE} y \ref{tb_coefs_psi_2EAE}.

\begin{table}
\begin{center}
\begin{small}
\begin{tabular}{|l|c|c|c|}
\hline
\textsc{Use of area} & $\psi_{0}$ & $\psi_{1}$ & $\psi_{2}$ \\
\hline
Domestic, residential areas & 0.7 & 0.5 & 0.3 \\
Office areas & 0.7 & 0.5 & 0.3 \\
Congregation areas  & 0.7 & 0.7 & 0.6 \\
Shopping areas & 0.7 & 0.7 & 0.6 \\
Storage areas & 1.0 & 0.9 & 0.8 \\
Traffic areas, weight of vehicle $\leq 30\ kN$ & 0.7 & 0.7 & 0.6 \\
Traffic areas, $30\ kN\ <$ weight of vehicle $\leq 160\ kN$ & 0.7 & 0.5 & 0.3 \\
Inaccessible Roofs  & 0.0 & 0.0 & 0.0 \\
\hline
\end{tabular}
\end{small}
\end{center}
\caption{Recommended values of $\Psi$ factors for buildings, according to EAE} \label{tb_coefs_psi_2EAE}
\end{table}

\begin{table}
\begin{center}
\begin{small}
\begin{tabular}{|l|c|c|c|}
\hline
\textsc{Climatic actions} & $\psi_{0}$ & $\psi_{1}$ & $\psi_{2}$ \\
\hline
\multicolumn{1}{|p{8cm}|}{Snow loads in buildings set over a thousand meters above sea level.} & 0.7 & 0.5 & 0.2 \\
\multicolumn{1}{|p{8cm}|}{Snow loads in buildings set under a thousand meters above sea level.} & 0.5 & 0.2 & 0.0 \\
Wind loads & 0.6 & 0.2 & 0.0 \\
Thermal action & 0.6 & 0.5 & 0.0 \\
\hline
\end{tabular}
\end{small}
\caption{Recommended values of $\Psi$ factors of simultaneity, according to EAE} \label{tb_coefs_psi_1EAE}
\end{center}
\end{table}

\subparagraph{According to IAP \cite{IAP}:} see table \ref{tb_coefs_psi_IAP}.

\begin{table}
\begin{center}
\begin{small}
\begin{tabular}{|l|c|c|c|}
\hline
\textsc{Variable actions} & $\psi_{0}$ & $\psi_{1}$ & $\psi_{2}$ \\
\hline
Traffic load model fatigue & 1.0 & 1.0 & 1.0 \\
Other variable actions & 0.6 & 0.5 & 0.2 \\
\hline
\end{tabular}
\end{small}
\caption{Values of $\Psi$ factors of simultaneity according to IAP.} \label{tb_coefs_psi_IAP}
\end{center}
\end{table}

\subsubsection{Calculation value $F_d$ of the actions} \label{sc_valor_calculo_acc}
The calculation value of an action is obtained by multiplying its characteristic value by the corresponding partial coefficient $\gamma_f$:

\begin{equation}
F_d= \gamma_f \cdot F_r
\end{equation}

The values of the coefficients $\gamma_f$ takes into account one or more of the following uncertainties:

\begin{enumerate}
\item uncertainties in the estimation of the representative value of the actions, in fact, the characteristic value is chosen admitting a 5\% probability of being exceeded during the working life of the structure;
\item uncertainties in the calculations results, due to simplifications in the models and to certain numeric errors (rounding, truncation, \ldots)
\item Uncertainty in the geometric and mechanical characteristics of the built structure. During the execution of the structure some errors can be committed \footnote{It is understood that these errors are within the tolerances established in the regulations} that can make the dimensions of the sections, the position of the reinforcement, the position of the axes, the mechanical characteristics of the materials, \ldots, be different from the theoretical.
\end{enumerate}

\paragraph{Values of the partial coefficients}
The coefficients $\gamma_f$ have different values in accordance with:
\begin{enumerate}
\item the limit state to be verified;
\item the design situation that is involved (see section \ref{sc_situaciones});
\item the variation of the action over time (according to classification in \ref{sc_var_tiempo});
\item the effect favourable o unfavourable of the action in the limit state that is verified;
\item the control level.
\end{enumerate}

\subparagraph{According to EHE:} the values of the partial coefficients $\gamma_f$ are specified in table \ref{tb_gf_ELS_EHE} for serviceability limit states and in table \ref{tb_gf_ELU_EHE} for ultimate limit states.

\begin{table}
\begin{center}
\begin{footnotesize}
\begin{tabular}{|l|c|c|}
\hline
\textsc{Action} & \multicolumn{2}{|c|}{\textsc{Effect}} \\
\hline
 & favourable & unfavourable \\
\hline
Permanent  & $\gamma_G= 1.00$ &  $\gamma_G= 1.00$ \\
\hline
Prestressing (pre-tensioned concrete) & $\gamma_{P}= 0.95$ &  $\gamma_{P}= 1.05$ \\
Prestressing (post-tensioned concrete) & $\gamma_{P}= 0.90$ &  $\gamma_{P}= 1.10$ \\ 
\hline
Permanent of a non-constant value & $\gamma_{G*}= 1.00$ &  $\gamma_{G*}= 1.00$ \\
\hline
Variable & $\gamma_Q= 0.00$ &  $\gamma_Q= 1.00$ \\
\hline
\multicolumn{3}{|l|}{\textsc{Notation:}} \\
\hline
\multicolumn{3}{|l|}{G: Permanent action.} \\
\multicolumn{3}{|l|}{P: Prestressing.} \\
\multicolumn{3}{|l|}{G*: Permanent action of a non-constant value.} \\
\multicolumn{3}{|l|}{Q: Variable action.} \\
\multicolumn{3}{|l|}{A: Accidental action.} \\
\hline
\end{tabular}
\end{footnotesize}
\caption{Partial factor for actions in serviceability limit states according to EHE.} \label{tb_gf_ELS_EHE}
\end{center}
\end{table}

\begin{table}
\begin{center}
\begin{footnotesize}
\begin{tabular}{|c|c|c|c|c|c|}
\hline
Action & Control level & \multicolumn{2}{|p{4cm}|}{Effect in persistent or transient design situations} &\multicolumn{2}{|p{4cm}|}{ Effect in accidental or seismic design situations} \\
\hline
 & & favourable & unfavourable & favourable & unfavourable \\
\hline
              & intense & $\gamma_G= 1.00$ &  $\gamma_G= 1.35$ & $\gamma_G= 1.00$ &  $\gamma_G= 1.00$ \\
G             & normal & $\gamma_G= 1.00$ &  $\gamma_G= 1.50$ & $\gamma_G= 1.00$ &  $\gamma_G= 1.00$ \\
              & low & $\gamma_G= 1.00$ &  $\gamma_G= 1.60$ & $\gamma_G= 1.00$ &  $\gamma_G= 1.00$ \\
\hline
              & intense & $\gamma_{G*}= 1.00$ &  $\gamma_{G*}= 1.50$ & $\gamma_{G*}= 1.00$ &  $\gamma_{G*}= 1.00$ \\
G*            & normal & $\gamma_{G*}= 1.00$ &  $\gamma_{G*}= 1.60$ & $\gamma_{G*}= 1.00$ &  $\gamma_{G*}= 1.00$ \\
              & low & $\gamma_{G*}= 1.00$ &  $\gamma_{G*}= 1.80$ & $\gamma_{G*}= 1.00$ &  $\gamma_{G*}= 1.00$ \\
\hline
              & intense & $\gamma_Q= 0.00$ &  $\gamma_Q= 1.50$ & $\gamma_Q= 0.00$ &  $\gamma_Q= 1.00$ \\
Q             & normal & $\gamma_Q= 0.00$ &  $\gamma_Q= 1.60$ & $\gamma_Q= 0.00$ &  $\gamma_Q= 1.00$ \\
              & low & $\gamma_Q= 0.00$ &  $\gamma_Q= 1.80$ & $\gamma_Q= 0.00$ &  $\gamma_Q= 1.00$ \\
\hline
A             & - & - & - &  $\gamma_A= 1.00$ &  $\gamma_A= 1.00$ \\
\hline
\multicolumn{6}{|l|}{\textsc{Notation:}} \\
\hline
\multicolumn{6}{|l|}{G: Permanent action.} \\
\multicolumn{6}{|l|}{G*: Permanent action of a non-constant value.} \\
\multicolumn{6}{|l|}{Q: Variable action.} \\
\multicolumn{6}{|l|}{A: Accidental action.} \\
\hline
\end{tabular}
\end{footnotesize}
\caption{Partial factor for actions in ultimate limit states according to EHE.} \label{tb_gf_ELU_EHE}
\end{center}
\end{table}

\subparagraph{According to EAE:} the values of the partial coefficients $\gamma_F$ to be used are specified int tables \ref{tb_gf_ELS_EAE} for serviceability limit states and in table \ref{tb_gf_ELU_EAE} for ultimate limit states.


\begin{table}
\begin{center}
\begin{footnotesize}
\begin{tabular}{|l|c|c|}
\hline
\textsc{Action} & \multicolumn{2}{|c|}{\textsc{Effect}} \\
\hline
 & favourable & unfavourable \\
\hline
Permanent  & $\gamma_G= 1.00$ &  $\gamma_G= 1.00$ \\
\hline
Permanent of a non-constant value & $\gamma_{G*}= 1.00$ &  $\gamma_{G*}= 1.00$ \\
\hline
Variable & $\gamma_Q= 0.00$ &  $\gamma_Q= 1.00$ \\
\hline
\end{tabular}
\end{footnotesize}
\caption{Partial factor for actions in serviceability limit states according to EAE.} \label{tb_gf_ELS_EAE}
\end{center}
\end{table}

\begin{table}
\begin{center}
\begin{footnotesize}
\begin{tabular}{|c|c|c|c|c|}
\hline
Action & \multicolumn{2}{|p{4cm}|}{Effect in persistent or transient design situations} &\multicolumn{2}{|p{4cm}|}{ Effect in accidental or seismic design situations} \\
\hline
 & favourable & unfavourable & favourable & unfavourable \\
\hline
G  & $\gamma_G= 1.00$ &  $\gamma_G= 1.35$ & $\gamma_G= 1.00$ &  $\gamma_G= 1.00$ \\
\hline
G* & $\gamma_{G*}= 1.00$ &  $\gamma_{G*}= 1.50$ & $\gamma_{G*}= 1.00$ &  $\gamma_{G*}= 1.00$ \\
\hline
Q  & $\gamma_Q= 0.00$ &  $\gamma_Q= 1.50$ & $\gamma_Q= 0.00$ &  $\gamma_Q= 1.00$ \\
\hline
A  & - & - &  $\gamma_A= 1.00$ &  $\gamma_A= 1.00$ \\
\hline
\multicolumn{5}{|l|}{\textsc{Notation:}} \\
\hline
\multicolumn{5}{|l|}{G: Permanent action.} \\
\multicolumn{5}{|l|}{G*: Permanent action of a non-constant value.} \\
\multicolumn{5}{|l|}{Q: Variable action.} \\
\multicolumn{5}{|l|}{A: Accidental action.} \\
\hline
\end{tabular}
\end{footnotesize}
\caption{Partial factor for actions in ultimate limit states according to EAE.} \label{tb_gf_ELU_EAE}
\end{center}
\end{table}

\subparagraph{According to IAP:} the values of the partial coefficients $\gamma_F$ to be used are specified int tables \ref{tb_gf_ELS_IAP} for serviceability limit states and in table \ref{tb_gf_ELU_IAP} for ultimate limit states.

\begin{table}
\begin{center}
\begin{footnotesize}
\begin{tabular}{|l|c|c|}
\hline
\textsc{Action} & \multicolumn{2}{|c|}{\textsc{Effect}} \\
\hline
 & favourable & unfavourable \\
\hline
Permanent  & $\gamma_G= 1.00$ &  $\gamma_G= 1.00$ \\
\hline
Internal prestressing (post-tensioned concrete) & $\gamma_{P_1}= 0.9$ &  $\gamma_{P_1}= 1.1$ \\
Internal prestressing (pre-tensioned concrete) & $\gamma_{P_1}= 0.95$ &  $\gamma_{P_1}= 1.05$ \\ 
\hline
External prestressing & $\gamma_{P_2}= 1.0$ &  $\gamma_{P_2}= 1.0$ \\
\hline
Other prestressing actions & $\gamma_{G*}= 1.00$ &  $\gamma_{G*}= 1.00$ \\
\hline
Rheological & $\gamma_{G*}= 1.00$ &  $\gamma_{G*}= 1.00$ \\
\hline
Thrust of the site & $\gamma_{G*}= 1.00$ &  $\gamma_{G*}= 1.00$ \\
\hline
Variable & $\gamma_Q= 0.00$ &  $\gamma_Q= 1.00$ \\
\hline
\multicolumn{3}{|l|}{\textsc{Notation:}} \\
\hline
\multicolumn{3}{|l|}{$G$: Permanent action.} \\
\multicolumn{3}{|l|}{$P_1$: Internal prestressing.} \\
\multicolumn{3}{|l|}{$P_2$: External prestressing.} \\
\multicolumn{3}{|l|}{$G*$: Permanent action of a non-constant value.} \\
\multicolumn{3}{|l|}{$Q$: Variable action.} \\
\multicolumn{3}{|l|}{$A$: Accidental action.} \\
\hline
\end{tabular}
\end{footnotesize}
\caption{Partial factor for actions in serviceability limit states according to IAP.} \label{tb_gf_ELS_IAP}
\end{center}
\end{table}

\begin{table}
\begin{center}
\begin{footnotesize}
\begin{tabular}{|c|c|c|c|c|}
\hline
Action & \multicolumn{2}{|p{4cm}|}{Effect in persistent or transient design situations} &\multicolumn{2}{|p{4cm}|}{Effect in accidental or seismic design situations} \\
\hline
 & favourable & unfavourable & favourable & unfavourable \\
\hline
Permanent  & $\gamma_G= 1.00$ &  $\gamma_G= 1.35$ & $\gamma_G= 1.00$ &  $\gamma_G= 1.00$ \\
\hline
Internal prestressing & $\gamma_{G*}= 1.00$ &  $\gamma_{G*}= 1.00$ & $\gamma_{G*}= 1.00$ &  $\gamma_{G*}= 1.00$ \\
\hline
External prestressing & $\gamma_{G*}= 1.00$ &  $\gamma_{G*}= 1.35$ & $\gamma_{G*}= 1.00$ &  $\gamma_{G*}= 1.00$ \\
\hline
Other prestressing actions & $\gamma_{G*}= 0.95$ &  $\gamma_{G*}= 1.05$ & $\gamma_{G*}= 1.00$ &  $\gamma_{G*}= 1.00$ \\
\hline
Rheological & $\gamma_{G*}= 1.0$ &  $\gamma_{G*}= 1.35$ & $\gamma_{G*}= 1.00$ &  $\gamma_{G*}= 1.00$ \\
\hline
Thrust of the site & $\gamma_{G*}= 1.00$ &  $\gamma_{G*}= 1.50$ & $\gamma_{G*}= 1.00$ &  $\gamma_{G*}= 1.00$ \\
\hline
Variable & $\gamma_Q= 0.00$ &  $\gamma_Q= 1.50$ & $\gamma_Q= 0.00$ &  $\gamma_Q= 1.00$ \\
\hline
Accidental & - & - &  $\gamma_A= 1.00$ &  $\gamma_A= 1.00$ \\
\hline
\end{tabular}
\end{footnotesize}
\caption{Partial factor for actions in ultimate limit states according to IAP.} \label{tb_gf_ELU_IAP}
\end{center}
\end{table}


\section{Design situations} \label{sc_situaciones}
Design situations, that take into account the circumstances under which the structure can be required during its execution and use, shall be classified as follows:

\begin{enumerate}
\item Persistent design situations, which refer to the conditions of normal use.
\item transient design situations, which refer to temporary conditions applicable to the structure, e.g. during execution or repair.
\item Accidental design situations, which refer to exceptional conditions applicable to the structure or to its exposure, e.g. to fire, explosion, impact or the consequences of localised failure.
%\item Seismic design situations, which refer to conditions applicable to the structure when subjected to seismic events.
\end{enumerate}

\section{Level of quality control}
A two level system for control during execution has been adopted:

\begin{itemize}
\item Intense control.
\item Normal control.
%\item Low control.
\end{itemize}

As will be seen later, the partial factors for a material or a member resistance depend on the level of inspection during construction.

\section{Limit states} \label{sc_el}
They can be defined as those states beyond which the structure no longer fulfils the relevant design criteria.

The design of the structure will be right when:

\begin{enumerate}
\item it is verified that no ultimate limit state is exceeded for the design situations and load cases defined in \ref{sc_comb_elu}, and
\item it is verified that no serviceability limit state is exceeded under the design situations and load cases defined in \ref{sc_comb_els}.
\end{enumerate}

\subsection{Ultimate limit states}
They are states associated with collapse or with other similar forms of structural failure. They generally correspond to the maximum load-carrying resistance of a structure or structural member.

The following ultimate limit states shall be verified where they are relevant:
- 
failure caused by fatigue or other time-dependent effects.


\begin{enumerate}
\item loss of equilibrium of the structure or any part of it, considered as a rigid body;
\item failure by excessive deformation, transformation of the structure or any part of it into a mechanism, rupture, loss of stability of the structure or any part of it, including supports and foundations;
\item failure caused by fatigue or other time-dependent effects.
\end{enumerate}

\subsection{Serviceability limit states}
They can be defined as states that correspond to conditions beyond which specified service requirements for a
structure or structural member are no longer met. These service requirements can concern:


\begin{itemize}
\item functionality.
\item comfort.
\item durability.
\item aesthetics.
\end{itemize}

The verification of serviceability limit states should be based on criteria concerning the following aspects :
\begin{enumerate}
\item deformations that affect:
\begin{itemize}
\item the appearance,
\item the comfort of users, or
\item the functioning of the structure (including the functioning of machines or services),
\end{itemize}
or that cause damage to finishes or non-structural members;

\item vibrations 
\begin{itemize}
\item that cause discomfort to people, or
\item that limit the functional effectiveness of the structure;
\end{itemize}
\item damage that is likely to adversely affect
\begin{itemize}
\item the appearance,
\item the durability, or
\item the functioning of the structure.
\end{itemize}
\end{enumerate}


\section{Combination of actions} \label{sc_comb}
When the verification of a structure is carried out by the partial factor method, it shall be verified than, in all relevant design situations, no relevant limit state is exceeded when design values for actions or effects of actions and resistances are used in the design models.

In order to eliminate the combinations that are not possible (or do not make sense), the following criteria will be considered:

\begin{itemize}
\item When an action is involved in a combination, none of its incompatible actions will be involved in that combination.
\item When an action is involved in a combination, all of its synchronous actions must be involved in that combination 
\footnote{See synchronous action and compatible action definitions in section \ref{sc_acc_rel_otras}.}
\end{itemize}
In what follows, we will consider any structure, under the following actions:
\begin{itemize}
\item $n_G$ permanent actions: $G_i$\footnote{The subscript refers to each of permanent actions on the structure $G_1$, $G_2$, $G_3$, $G_4$, \ldots, $G_{n_G}$ }.
\item $n_{G*}$ permanent actions of a non-constant value: $G*_j$.
\item $n_Q$ variable actions: $Q_l$.
\item $n_A$ accidental actions: $Q_m$.
\item $n_{AS}$ seismic actions: $Q_n$.
\end{itemize}


\subsection{Combinations of actions for ultimate limit states} \label{sc_comb_elu}
For the selected design situations and the relevant ultimate limit states the individual actions for the critical load cases should be combined as detailed in this section.

\subsubsection{Combinations of actions for persistent or transient design situations} \label{sc_comb_elu_spt}
For each variable action, a group of combinations with this action as \emph{leading variable action} will be considered \footnote{See section \ref{sc_modo_partic_acc}.}.

\begin{equation} \label{eq_comb_spt}
\sum_{i=1}^{n_G} \gamma_G \cdot G_{k,i} +\sum_{j=1}^{n_{G*}} \gamma_{G*} \cdot G*_{k,j} + \gamma_Q \cdot Q_{k,d} + \sum_{l=1}^{d-1} \gamma_Q \cdot Q_{r0,l} + \sum_{l=d+1}^{n_Q} \gamma_Q \cdot Q_{r0,l} 
\end{equation}

\noindent where:

\begin{description}
\item{$\gamma_G \cdot G_{k,i}$:} design value of the permanent action $i$, obtained from its characteristic value  ;
\item{$\gamma_{G*} \cdot G*_{k,j}$:} design value of the permanent action of a non-constant value $j$, obtained from its characteristic value;
\item{$\gamma_Q \cdot Q_{k,d}$:} design value of the leading variable action $d$, obtained from its characteristic value;
\item{$\gamma_Q \cdot Q_{r0,l}$:} design value of la variable action $l$, obtained from its accompanying value.
\end{description}

\paragraph{Number of combinations to be considered:} According to section \ref{sc_valor_calculo_acc}:

\begin{itemize}
\item The permanent actions, in ULS combinations corresponding to persistent or transient design situations, will have two non-zero partial factors.
\item In the same case, the permanent actions of a non-constant value will have two non-zero partial factors that, in some cases, can be equal (see the case of internal prestressing on the table \ref{tb_gf_ELU_IAP}).
\item The variable actions will have a single non-zero partial factor.
\end{itemize}

\noindent therefore, assuming that:

\begin{description}
\item{$n_{G2}$} is the number of permanent actions that have two different partial factors;
\item{$n_{G1}$} is the number of permanent actions that have a single partial factor\footnote{Because both factors are equal.};
\item{$n_{G*2}$} is the number of permanent actions of a non-constant value that have two different partial factors;
\item{$n_{G*1}$} the number of permanent actions of a non-constant value that have a single partial factor, and
\item $n_{Q}$ is the number of variable actions, all of then have a single partial factor.
\end{description}
If, by now, incompatibility or synchronicity of actions is ignored, for each variable action we'll have:

\begin{itemize}
\item $2^{n_{G2}}$ combinations of permanent actions in the set $G2$;
\item 1 combination of permanent actions in the set $G1$;
\item $2^{n_{G*2}}$ combinations of permanent actions in the set $G*2$;
\item 1 combination of permanent actions in the set $G*1$, and
\item $2^{n_{Q}-1}$ combinations of accompanying variable actions.
\end{itemize}

As, for each leading action two partial factors must be considered, the total number of combinations $n_{comb,spt}$ for persistent or transient design situations will be equal to the cartesian product of the previous combinations by  $2^{n_{Qd}}$, where $Qd$ is the number of variable actions that can be leading:

\begin{equation} \label{eq_ncomb_spt}
n_{comb,ULS,spt}= 2^{n_{G2}} \cdot 2^{n_{G*2}} \cdot 2^{n_{Q}-1} \cdot 2^{n_{Qd}}= 2^{n_{G2}+n_{G*2}+n_{Q}+n_{Qd}-1}
\end{equation}

Among these combinations, those that are incompatibles must be eliminated. 


For synchronic actions, the following procedure can be followed:

Let $a$ be a synchronic action of the action $b$:

\begin{enumerate}
\item $a$ is eliminated from the list of variable actions;
\item the action $a+b$ is added to the list of variable actions;
\item incompatibility between $a+b$ and $b$ actions is set.
\end{enumerate} 


\subsubsection{Combinations of actions for accidental design situations}
For each variable action $Q_l$, $n_A$ combinations with that action as leading are formed.

\begin{equation}\label{eq_comb_acc}
\sum_{i=1}^{n_G} \gamma_G \cdot G_{k,i} +\sum_{j=1}^{n_{G*}} \gamma_{G*} \cdot G*_{k,j} + A_{k,m} + \gamma_Q \cdot Q_{r1,d} + \sum_{l=1}^{d-1} \gamma_Q \cdot Q_{r2,l} + \sum_{l=d+1}^{n_Q} \gamma_Q \cdot Q_{r2,l} 
\end{equation}

\noindent where:

\begin{description}
\item{$A_{k,m}$:} design value of the accidental action $m$, obtained from its characteristic value;
\item{$\gamma_Q \cdot Q_{r1,d}$:} design value of the leading variable action $d$, obtained from its representative frequent value;
\item{$\gamma_Q \cdot Q_{r2,l}$:} design value of la variable action $l$, obtained from its representative quasi-permanent value.
\end{description}

\paragraph{Number of combinations to be considered:} it results the same number of combinations for each sum than in the case solved in the paragraph \ref{sc_comb_elu_spt} (see \ref{eq_ncomb_spt} expression), though, in this case, the representative values of the variable actions are other ones. If, as usual, the partial factors for seismic actions are equal for favourable and unfavourable actions, it suffices to multiply by the number of accidental actions $n_A$.

\begin{equation} \label{eq_ncomb_acc}
n_{comb,ULS,acc}= 2^{n_{G2}+n_{G*2}+n_{Q}+n_{Qd}-1} \cdot n_A
\end{equation}

For incompatible actions, the procedure provided for in section \ref{sc_comb_elu_spt} is applicable.


\subsubsection{Combinations of actions for seismic design situations} \label{sc_comb_elu_sism}
For each seismic action one combination will be formed:

\begin{equation}\label{eq_comb_sis}
\sum_{i=1}^{n_G} \gamma_G \cdot G_{k,i} +\sum_{j=1}^{n_{G*}} \gamma_{G*} \cdot G*_{k,j} + AS_{k,n} + \sum_{l=1}^{n_Q} \gamma_Q \cdot Q_{r2,l}
\end{equation}

\noindent where:

\begin{description}
\item{$A_{k,m}$} is the design value of the accidental action $m$, and
\item{$\gamma_Q \cdot Q_{r2,l}$} is the design value of the variable action $l$, obtained from its representative quasi-permanent value.
\end{description}

\paragraph{Number of combinations to be considered:} 

\begin{equation} \label{eq_ncomb_sism}
n_{comb,ULS,sism}= 2^{n_{G2}+n_{G*2}+n_{Q}} \cdot n_{AS}
\end{equation}

For incompatible actions, the procedure provided for in section \ref{sc_comb_elu_spt} is applicable.

\subsection{Combinations of actions for serviceability limit states} \label{sc_comb_els}
For the selected design situations and the relevant serviceability limit states the individual actions for the critical load cases should be combined as detailed in this section.


\subsubsection{Rare combinations:}\label{sc_comb_els_pf}
For each variable action, one combination with this action as \emph{leading variable action} will be considered.

\begin{equation}
\sum_{i=1}^{n_G} G_{k,i} + \sum_{j=1}^{n_{G*}} G*_{k,j} + Q_{k,d} + \sum_{l=1}^{d-1} Q_{r0,l} + \sum_{l=d+1}^{n_Q} Q_{r0,l} 
\end{equation}

In a general case, with no incompatible or concomitant combinations, the following combinations will be formed (see notation in section \ref{sc_comb_elu_spt}):

\begin{equation} \label{eq_ncomb_els_pf}
n_{comb,SLS,pf}= 2^{n_{G2}+n_{G*2}+n_{Q}+n_{Qd}-1}
\end{equation}

Since the partial factors are for serviceability limit states, the sets $G2$ y $G*2$ generally will not match those for ultimate limit states. Given that in many cases both partial factors are equal to the unity, the cardinality of these sets will be much lower than the equivalent in paragraph \ref{sc_comb_elu_spt}.


For incompatible actions, the procedure provided for in section \ref{sc_comb_elu_spt} is applicable.

\subsubsection{Frequent combinations:}\label{sc_comb_els_f}
For each variable action, one combination in which this action acts as \emph{leading} will be formed.

\begin{equation}
\sum_{i=1}^{n_G} G_{k,i} + \sum_{j=1}^{n_{G*}} G*_{k,j} + Q_{r1,d} + \sum_{l=1}^{d-1} Q_{r2,l} + \sum_{l=d+1}^{n_Q} Q_{r2,l} 
\end{equation}

the number of combinations will be the same as the precedent case, since only the combination factors can vary.

\subsubsection{Quasi-permanent combinations:}\label{sc_comb_els_cp}

\begin{equation}
\sum_{i=1}^{n_G} G_{k,i} + \sum_{j=1}^{n_{G*}} G*_{k,j} + \sum_{l=1}^{n_Q} Q_{r2,l} 
\end{equation}

the number of combinations will be:

\begin{equation} \label{eq_ncomb_els_cp}
n_{comb,SLS,cp}= 2^{n_{G2}+n_{G*2}+n_{Q}}
\end{equation}

\subsection{Combinations to be considered in the calculation}
According to the discussion in the previous sections, the number of combinations for a general calculations will be the following:
\begin{center}
\begin{small}
\begin{tabular}{lr}
\hline
Ultimate limit states & number of combinations \\
\hline
Persistent or transient design situations & $2^{(n_G+n_{G*}+n_Q)}\cdot n_Q$ \\
Accidental design situations & $2^{(n_G+n_{G*}+n_Q)} \cdot n_Q \cdot n_A$ \\
Seismic design situations & $2^{(n_G+n_{G*}+n_Q)}\cdot n_{AS} $ \\
\hline
Total ULS & $2^{(n_G+n_{G*}+n_Q)} \cdot (n_Q (1+n_A)+n_{AS})$ \\ 
\hline
Serviceability limit states & \\
\hline
Rare combinations & $n_Q$ \\
Frequent combinations & $n_Q$ \\
Quasi-permanent combination & 1 \\
\hline
Total SLS & $2 n_Q + 1$ \\
\hline
\textbf{Total combinations} & $\mathbf{2^{(n_G+n_{G*}+n_Q)} \cdot (n_Q (1+n_A)+n_{AS}) +2 n_Q + 1}$ \\ 
\hline
\end{tabular}
\end{small}
\end{center}

For example, if we had:

\begin{itemize}
\item $2$ permanent actions;
\item $1$ permanent action of a non-constant value;
\item $3$ variable actions;
\item $1$ accidental action, and
\item $2$ seismic actions
\end{itemize}

\noindent the number of combinations will be:

\begin{center}
\begin{small}
\begin{tabular}{lr}
\hline
Ultimate limit states & number of combinations \\
\hline
Persistent or transient design situations & $2^{(2+1+3)}\times 3= 192$ \\
Accidental design situations & $2^{(2+1+3)}\times 3 \times 1= 192$ \\
Seismic design situations & $2^{(2+1+3)}\times 2= 128$ \\
\hline
Total ULS & $2^{(2+1+3)} \times (3\times(1+1)+2)= 512$ \\ 
\hline
Serviceability limit states & \\
\hline
Rare combinations & $3$ \\
Frequent combinations & $3$ \\
Quasi-permanent combination & $1$ \\
\hline
Total SLS & $6 + 1= 7$ \\
\hline
\textbf{Total combinations} & $\mathbf{519}$ \\ 
\hline
\end{tabular}
\end{small}
\end{center}

\subsection{Algorithm to write the complete list of combinations}

\subsubsection{Combinations for ultimate limit states}
Each of the sums in expressions (\ref{eq_comb_spt}),(\ref{eq_comb_acc}) y (\ref{eq_comb_sis}) appears as follows:


\begin{equation} \label{eq_sumatorio}
\sum_{i=1}^n \gamma_f \cdot F_{r,i}
\end{equation}

For each action $F_i$ the partial factor can take two values, depending on the effect favourable or unfavourable of the action\footnote{We assume a priory unknown the effect favourable or unfavourable of the action for the limit state and structural element in analysed} .

The design value of the action $F_{r,i}$ depends on:

\begin{itemize}
\item its variation in time (G,G*,A,A,AS);
\item its role in the combination, as leading or accompanying action;
\item if there is or not accidental actions in the combination;
\item the nature of the action (climatic or live loads).
\end{itemize}

\noindent in any case, for any combination, the value of $F_{r,i}$ is known.

Moreover, the value of \emph{n} is known for each sum.

Following this, the summands of (\ref{eq_sumatorio}) correspond to the variations with repetition \footnote{The variations with repetition of \emph{n} elements taken \emph{k} by \emph{k} are the arranged groups formed by k elements from A (which may be repeated)} of two elements 
\footnote{The partial factors corresponding to favourable and unfavourable effects} taken \emph{n} by \emph{n}.

To write the variations with repetition of expression (\ref{eq_sumatorio}), proceed as follows:

Let $\mathbf{\gamma_f}_v$ be the row vector whose components are the partial factors of the variation $v$ ($1 \leq v \leq 2^n$):

\begin{equation}
\mathbf{\gamma_f}_v= [\gamma_{f,1}, \gamma_{f,2}, \cdots, \gamma_{f,i}, \cdots, \gamma_{f,n}]
\end{equation}

\noindent that's to say, the element $\gamma_{f,i}$ is the partial factor (favourable or unfavourable) of action $F_{r,i}$.

Let $\mathbf{F_r}$ be the column vector whose components are the actions $F_{r,i}$ of the expression (\ref{eq_sumatorio}):

\begin{equation}
\mathbf{F_r}^T= [F_{r,1}, F_{r,2}, \cdots, F_{r,i}, \cdots, F_{r,n}]
\end{equation}

\noindent then, the expression (\ref{eq_sumatorio}) is equivalent to the scalar product:

\begin{equation}
\sum_{i=1}^n \gamma_f \cdot F_{r,i}= \mathbf{\gamma_f}_v \cdot \mathbf{F_r}
\end{equation}

\noindent and it must be formed as many scalar products as variations with repetition can be arranged, that's to say, $2^n$.

Let $S_{F,v}$ be the sum that corresponds to variation \emph{v},

\begin{equation}
S_{F_r,v}= \mathbf{\gamma_f}_v \cdot \mathbf{F_r}
\end{equation}

then each of sums (\ref{eq_comb_spt}),(\ref{eq_comb_acc}) and (\ref{eq_comb_sis}) gives rise to set of variations:

\begin{align} \notag
S_{F_r,1} &= \mathbf{\gamma_f}_1 \cdot \mathbf{F_r} \\ \notag
S_{F_r,2} &= \mathbf{\gamma_f}_2 \cdot \mathbf{F_r} \\ \notag
\cdots & \\ \notag
S_{F_r,v} &= \mathbf{\gamma_f}_v \cdot \mathbf{F_r}\\ \notag
\cdots & \\ \notag
S_{F_r,n_F} &= \mathbf{\gamma_f}_{n_F} \cdot \mathbf{F_r}
\end{align}

\noindent where $n_F$ is the number of actions in each case, that's to say $n_G,\ n_{G*},\ n_Q,\ n_A,$ or $n_{AS}$.

Therefore, the summands (\ref{eq_comb_spt}),(\ref{eq_comb_acc}) and (\ref{eq_comb_sis}) will be one of the following scalar products:

\begin{itemize}
\item Summand corresponding to permanent actions: $S_{G_r,v_G}$ ($1 \leq v_G \leq 2^{n_G}$).
\item Summand corresponding to permanent actions of a non-constant value: $S_{G*_r,v_{G*}}$ ($1 \leq v_{G*} \leq 2^{n_{G*}}$).
\item Summand corresponding to variable actions: $S_{Q_r,v_Q}$ ($1 \leq v_Q \leq 2^{n_Q}$). 
\item Summand corresponding to accidental actions: $S_{A_r,v_A}$ ($1 \leq v_A \leq 2^{n_A}$). 
\item Summand corresponding to seismic actions: $S_{AS_r,v_{AS}}$ ($1 \leq v_{AS} \leq 2^{n_{AS}}$).
\end{itemize}

\paragraph{Combinations of actions for persistent or transient design situations}
With this notation, the expression (\ref{eq_comb_spt}) can be written as follows:

\begin{equation}
CQ_{v_G,v_{G*},v_Q,d}= S_{G_k,v_G}+S_{G*_k,v_{G*}}+S_{Q_{r0,d},v_Q}
\end{equation}

\noindent where:
\begin{description}
\item{$v_G$} is the variation corresponding to the permanent actions;
\item{$v_{G*}$} is the variation corresponding to the permanent actions of a non-constant value;
\item{$v_{Q}$} is the variation corresponding to the variable actions;
\item{$d$} is the index that corresponds to the leading variable action, and
\item{$\mathbf{Q}_{r0,d}$} is the vector $[Q_{r0,1}, Q_{r0,2}, \cdots, Q_{r0,d-1},\ Q_{k,d},\ Q_{r0,d+1}, \cdots, Q_{r0,n_Q}]$
\end{description}

\paragraph{Combinations of actions for accidental design situations}
Similarly, the expression (\ref{eq_comb_acc}) can be written as follows:

\begin{equation}
CA_{v_G,v_{G*},v_Q,d,m}= S_{G_k,v_G}+S_{G*_k,v_{G*}}+S_{Q_{r2,d},v_Q}+ A_{k,m}
\end{equation}

\noindent where:
\begin{description}
\item{$v_G$} is the variation corresponding to the permanent actions;
\item{$v_{G*}$} is the Variation corresponding to the permanent actions of a non-constant value;
\item{$v_{Q}$} is the variation corresponding to the variable actions;
\item{$d$} is the index corresponding to the leading variable action;
\item{$\mathbf{Q}_{r2,d}$} is the vector $[Q_{r2,1}, Q_{r2,2}, \cdots, Q_{r2,d-1},\ Q_{r1,d},\ Q_{r2,d+1}, \cdots, Q_{r2,n_Q}]$;
\item{$m$} is the index that corresponds to the accidental action considered, and
\item{$A_{k,m}$} is the design value of the accidental action $m$.
\end{description}

\paragraph{Combinations for seismic design situations}
Similarly, the expression (\ref{eq_comb_sis}) can be written as follows:

\begin{equation}
CS_{v_G,v_{G*},v_Q,n}= S_{G_k,v_G}+S_{G*_k,v_{G*}}+S_{Q_{r2},v_Q}+ AS_{k,n}
\end{equation}

\noindent where
\begin{description}
\item{$v_G$} is the variation corresponding to the permanent actions;
\item{$v_{G*}$} is the variation corresponding to the permanent actions of a non-constant value;
\item{$v_{Q}$} is the variation corresponding to the variable actions;
\item{$\mathbf{Q}_{r2}$} is the vector $[Q_{r2,1}, Q_{r2,2}, \cdots, Q_{r2,n_Q}]$;
\item{$n$} is the index of the seismic action considered, and
\item{$AS_{k,n}$} is the design value of the seismic action $n$.
\end{description}

\paragraph{Calculation algorithm}
The proposed algorithm for writing all the combinations for ultimate limit states is as follows:

\begin{enumerate}
\item calculation of all the variations corresponding to actions G: $\mathbf{\gamma}_{g,v_G}$ ($1 \leq v_G \leq 2^{n_G}$)
\item calculation of all the variations corresponding to actions G*: $\mathbf{\gamma}_{g*,v_{G*}}$ ($1 \leq v_{G*} \leq 2^{n_{G*}}$)
\item calculation of all the variations corresponding to actions Q: $\mathbf{\gamma}_{q,v_Q}$ ($1 \leq v_Q \leq 2^{n_Q}$)
\item from $d=1$ to $d=n_q$
  \begin{enumerate}
  \item calculation of all the combinations $CQ_{v_G,v_{G*},v_Q,d}$. \label{paso_CQ}
  \end{enumerate}
\item from $d=1$ to $d=n_Q$
  \begin{enumerate}
  \item from $m=1$ to $m=n_A$
    \begin{enumerate}
      \item calculation of all the combinations $CA_{v_G,v_{G*},v_Q,d,m}$. \label{paso_CA}
    \end{enumerate}
  \end{enumerate}
  \item from $n=1$ to $n=n_{AS}$
    \begin{enumerate}
      \item calculation of all the combinations $CS_{v_G,v_{G*},v_Q,n}$. \label{paso_CS}
    \end{enumerate}
\item end
\end{enumerate}

\noindent refinement of step \ref{paso_CQ}:
\begin{enumerate}
\item from $v_G=1$ to $v_G=2^{n_G}$
\begin{enumerate}
\item calculate $S_{G_k,v_G}$
\item from $v_{G*}=1$ to $v_{G*}=2^{n_{G*}}$
\begin{enumerate}
\item calculate $S_{G*_k,v_{G*}}$
\item from $v_Q=1$ to $v_Q=2^{n_Q}$
\begin{enumerate}
\item calculate $S_{Q_{r0,d},v_Q}$
\item calculate $CQ_{v_G,v_{G*},v_Q,d}= S_{G_k,v_G}+S_{G*_k,v_{G*}}+S_{Q_{r0,d},v_Q}$
\end{enumerate}
\end{enumerate}
\end{enumerate}
\item end
\end{enumerate}

\noindent refinement of step \ref{paso_CA}:
\begin{enumerate}
\item from $v_G=1$ to $v_G=2^{n_G}$
\begin{enumerate}
\item calculate $S_{G_k,v_G}$
\item from $v_{G*}=1$ to $v_{G*}=2^{n_{G*}}$
\begin{enumerate}
\item calculate $S_{G*_k,v_{G*}}$
\item from $v_Q=1$ to $v_Q=2^{n_Q}$
\begin{enumerate}
\item calculate $S_{Q_{r2,d},v_Q}$.
\item calculate $CA_{v_G,v_{G*},v_Q,d,m}= S_{G_k,v_G}+S_{G*_k,v_{G*}}+S_{Q_{r2,d},v_Q}+A_{k,m}$
\end{enumerate}
\end{enumerate}
\end{enumerate}
\item end
\end{enumerate}

\noindent refinement of step \ref{paso_CS}:
\begin{enumerate}
\item from $v_G=1$ to $v_G=2^{n_G}$
\begin{enumerate}
\item calculate $S_{G_k,v_G}$
\item from $v_{G*}=1$ to $v_{G*}=2^{n_{G*}}$
\begin{enumerate}
\item calculate $S_{G*_k,v_{G*}}$
\item from $v_Q=1$ to $v_Q=2^{n_Q}$
\begin{enumerate}
\item calculate $S_{Q_{r2},v_Q}$.
\item calculate $CS_{v_G,v_{G*},v_Q,n}= S_{G_k,v_G}+S_{G*_k,v_{G*}}+S_{Q_{r2},v_Q}+AS_{k,n}$
\end{enumerate}
\end{enumerate}
\end{enumerate}
\item end
\end{enumerate}

\subsubsection{Combinations for serviceability limit states}
Taking into account the partial factors for serviceability limit states, if:

\begin{equation}
S_{G_k}= \sum_{i=1}^{n_G} G_{k,i}
\end{equation}

\begin{equation}
S_{G*_k}= \sum_{j=1}^{n_{G*}} G*_{k,j}
\end{equation}

\begin{equation}
S_{Q_{r0},d}= \sum_{l=1}^{d-1} Q_{r0,l}+Q_{k,d}+\sum_{l=d+1}^{n_Q} Q_{r0,l}
\end{equation}

\begin{equation}
S_{Q_{r2},d}= \sum_{l=1}^{d-1} Q_{r2,l}+Q_{r1,d}+\sum_{l=d+1}^{n_Q} Q_{r2,l}
\end{equation}

\noindent and
\begin{equation}
S_{Q_{r2}}= \sum_{l=1}^{n_Q} Q_{r2,l}
\end{equation}
\noindent then:
\noindent the $n_Q$ rare combinations will be:

\begin{equation}
CPF_{d}= S_{G_k}+S_{G*_k}+S_{Q_{r0},d}
\end{equation}

\noindent the $n_Q$ frequent combinations will be:

\begin{equation}
CF_{d}= S_{G_k}+S_{G*_k}+S_{Q_{r2},d}
\end{equation}

\noindent and the quasi-permanent combination will be:

\begin{equation}
CCP= S_{G_k}+S_{G*_k}+S_{Q_{r2}}
\end{equation}

\paragraph{Calculation algorithm}
The calculation algorithm of all the combinations for serviceability limit states would be expressed as follows:

\begin{enumerate}
\item calculation of $S_{G_k}$
\item calculation of $S_{G*_k}$
\item from $d=1$ to $d=n_Q$
  \begin{enumerate}
  \item calculate $S_{Q_{r0},d}$
  \item calculate $CPF_d= S_{G_k}+S_{G*_k}+S_{Q_{r0},d}$
  \end{enumerate}
\item from $d=1$ to $d=n_Q$
  \begin{enumerate}
  \item calculate $S_{Q_{r2},d}$
  \item calculate $CF_d= S_{G_k}+S_{G*_k}+S_{Q_{r2},d}$
  \end{enumerate}
\item calculation of $S_{Q_{r2}}$
\item calculate $CCP= S_{G_k}+S_{G*_k}+S_{Q_{r2}}$
\item end
\end{enumerate}






