\chapter{Finite element model components}

\section{Introduction}
XC is comprised of a set of Python modules and objects to perform:

\begin{itemize}
\item creation of the finite element model,
\item specification of an analysis procedure,
\item selection of quantities to be monitored during the analysis,
\item and the output of results.
\end{itemize}

In each finite element analysis, an analyst constructs 5 main types of objects, as shown in figure~\ref{main}:

\begin{figure}[htpb]
%% \begin{center}
%% \leavevmode
%% \hbox{%
%% %\epsfxsize=6.0in
%% %\epsfysize=4.2in
%% \epsffile{./main_objects.eps}}
%% \end{center}
\caption{Main Objects in an Analysis}
\label{main}
\end{figure}

Those main objects are:
\begin{enumerate}
\item {\bf Preprocessor}: As in any finite element analysis, the analyst's first step is to subdivide the body under study into elements and nodes, to define loads acting on the elements and nodes, and to define constraints acting on the nodes. The Preprocessor is the object in the program responsible for building the Element, Node, LoadPattern, TimeSeries, Load and Constraint objects.

\item {\bf Domain}: The Domain object is responsible for storing the objects created by the Preprocessor object and for providing the Analysis and Recorder objects access to these objects.

\item {\bf Analysis}: Once the analyst has defined the model, the next step is to define the analysis that is to be performed on the model. This may vary from a simple static linear analysis to a transient non-linear analysis. The Analysis object is responsible for performing the analysis. In XC each Analysis object is composed of several component objects, which define how the analysis is performed. The component classes consist of the following: { SolutionAlgorithm}, {Integrator}, { ConstraintHandler}, { DOF\_Numberer}, {SystemOfEqn}, {Solver}, and {AnalysisModel}. 

\item {\bf Recorder}: Once the model and analysis objects have been defined, the analyst has the option of specifying what is to be monitored during the analysis. This, for example, could be the displacement history at a node in a transient analysis or the entire state of the model at each step in the solution procedure. Several Recorder objects are created by the analyst to monitor the analysis.
  \item {\bf Post-processor}: Postprocessing may be defined as the “art of results representation”. The post-processor is composed by the objects an modules that organize the output of the analysis such that it is easily understandable by the user. It can include checks on the codes and standards to which the construction must comply.
\end{enumerate}


\section{Nodes}

\subsection{Description}
The nodes of a finite element mesh are the points where the degrees of freedom reside. Each node object has, at least, the following information:

\begin{itemize}
\item Coordinates wich define its position in space. Typically (x,y,z) coordinates.
\item Definition of the degrees of freedom in the node (displacements, rotations,\ldots)
\end{itemize}

The nodes can also serve to define loads or masses that act over the model at its position.

\subsection{Node creation}

To create a node you can use the following commands:

\begin{lstlisting}[frame=single]
  nodos.newNodeXY(x,y)
  nodos.newNodeIDXY(tag,x,y)
  nodos.newNodeXYZ(x,y,z)
  nodos.newNodeIDXYZ(x,y,z)
\end{lstlisting}

\noindent where:

\begin{description}
\item{nodos:} is a node container obtained from the modeler.
\item{tag:} is an integer that identifies the node in the model.
\item{(x,y) or (x,y,z):} are the cartesian coordinates that define node's position.
\end{description}


\section{Constraints}
In a finite element problem, the boundary conditions are the specified values of the field variables (displacement, rotations, pore pressures,\ldots).

\subsection{Essential and natural boundary conditions}
Essential boundary conditions are conditions that are imposed explicitly on the solution and natural boundary conditions are those that automatically will be satisfied after solution of the problem.

The former class of boundary conditions involve one or more degrees of freedom and are imposed by manipulating the left hand side (LHS) of the system of equations (the side of the stiffness matrix).

The natural boundary conditions are imposed by manipulating the right hand side (RHS) of the system of equations (the side of the force vector).



\subsubsection{Multi-point constraints}\label{sc_mp_constraints}

Multipoint constraints are used to impose linear relationships between some of the degrees of freedom of the model as in:

\begin{equation}
  \sum_{i} A_i u_i= 0
\end{equation}

\noindent where $A_i$ are constant factors and $u_i$ are degrees of freedom of the model.

This type of constraint allows considerable freedom in describing relations between degrees of freedom. They are used for example to create rigid elements and to link a node to an element.

\paragraph{Description}
An MP\_Constraint represents a multiple point constraint in the domain. A multiple point constraint imposes a relationship between the displacement for certain dof at two nodes in the model, typically called the {\em retained} node and the {\em constrained} node:

\begin{equation}
U_c = C_{cr} U_r
\end{equation}


An MP\_Constraint is responsible for providing information on the relationship between the dof, this is in the form of a constraint matrix, $C_{cr}$, and two ID objects, {\em retainedID} and {\em constrainedID} indicating the dof's at the nodes represented by $C_{cr}$. For example, for the following constraint imposing a relationship between the displacements at node $1$, the constrained node, with the displacements at node $2$, the retainednode in a problem where the x,y,z components are identified as the 0,1,2 degrees-of-freedom:

\begin{align}
u_{1,x} &= 2 u_{2,x} + u_{2,z} \\
u_{1,y} &= 3 u_{2,z}
\end{align}

\noindent the constraint matrix is:

\begin{equation}
C_{cr} =
\left[
\begin{array}{cc}
2 & 1  \\
0 & 3
\end{array}
\right] 
\end{equation}

\noindent and the vectors defining the dof's at the nodes are:

\begin{align}
constrainedID &= [0, 1] \\
retainedID &= [0, 2]
\end{align}



